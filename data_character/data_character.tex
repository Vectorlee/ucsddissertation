\chapter{Threat Intelligence Data Characteristic}

\section{Introduction}

While each organization naturally collects a certain amount of threat
intelligence data on its own (e.g., the attacks they repel, the e-mail
spam they filter, etc.) any single entity has a limited footprint and
few are instrumented to carefully segregate crisp signals of attacks
from the range of ambiguity found in normal production network and
system logs.  Thus, it is now commonly accepted that threat
intelligence data procurement is a specialized activity whereby
third-party firms, and/or collections of public groups, employ a range
of monitoring techniques to aggregate, filter and curate quality
information about current threats.  Indeed, the promised operational
value of threat intelligence has created a thriving (multi-billion
dollar) market~\cite{timarket}. Established security companies with
roots in anti-virus software or network intrusion detection now offer
threat intelligence for sale, while some vendors specialize in threat
intelligence exclusively, often promising coverage of more
sophisticated threats than conventional sources.

Unfortunately, in spite of this tremendous promise, there has been
little empirical assessment of threat intelligence data or even a
consensus about what such an evaluation would entail.  Thus, consumers
of \ti\ products have limited means to compare offerings
or to factor the cost of such products into any model of the benefit
to operational security that might be offered.

This issue motivates our work to provide a grounded,
empirical footing for addressing such questions.  In particular, this
paper makes the following contributions:
\begin{prettylist}
\item We introduce a set of basic \emph{threat intelligence metrics}
and describe a methodology for measuring them, notably: \veryemph{Volume},
\veryemph{Differential Contribution}, \veryemph{Exclusive Contribution},
\veryemph{Latency}, \veryemph{Coverage} and \veryemph{Accuracy}.
\item We analyze \numipfeeds\ distinct IP address \ti\ sources covering
six categories of threats and \numhashfeeds\ distinct malware file hash
\ti\ sources, and report their metrics.
\item We demonstrate techniques to evaluate the accuracy and coverage of
certain categories of \ti\ sources.
\item We conduct the analyses in two different time periods two years apart,
and demonstrate the strong consistency between the findings.
\end{prettylist}

From our analysis, we find that while a few \ti\ data sources show
significant overlap, most do not.  This result is consistent with the
hypothesis advanced by~\cite{thomas2016abuse} that different kinds of
monitoring infrastructure will capture different kinds of attacks, but
we have demonstrated it in a much broader context.  We also reveal
that underlying this issue are broader limitations of \ti\ sources in
terms of coverage (most indicators are unique) and accuracy (false
positives may limit how such data can be used operationally).
Finally, we present a longitudinal analysis suggesting that these
findings are consistent over time.
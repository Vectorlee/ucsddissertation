\subsection{Intersection and Exclusive Contribution}
\label{sec:hash-overlap}

As I mentioned earlier, to conduct intersection and exclusive
analysis of file hash feeds, I need to convert indicators into the
same hash type. Here I convert non-MD5 hashes into MD5s, using either
metadata in the indicator itself (i.e., if it reports values for
multiple hash functions) or by querying the source hash from
VirusTotal~\cite{VirusTotal} which reports the full suite of hashes
for all files in its dataset.  However, for a small fraction of hashes
I am unable to find aliases to conver them to the MD5 representation
and must exclude them from the analysis in this section.  This filtering is
reflected in Table~\ref{tab:md5-volume-1}, in which the Volume column
represents the number of unique hashes found in each feed and the Converted
column is the subset that I have been able to normalize to a MD5
representation.

\finding\ The intersections between hash feeds are minimal,
even among the feeds that have multiple orders of magnitude differences in size.
Across all feeds, only PA Analyst has relatively high intersections: PA Analyst
shares 27\% of PA OSINT's MD5s and 13\% of PA Twitter Emotet's MD5s. PA Malware
Indicators has a small intersection also with these two feeds. All other
intersections are around or less than 1\%. Consequently, the vast majority of
MD5s are unique to one feed, as recorded in column \textit{Exclusive} in
Table~\ref{tab:md5-volume-1}. The ``lowest'' exclusivity belongs to PA Twitter
Emotet and PA OSINT (still 77.29\% and 71.65\%, respectively). All other feeds
showcase an over 95\% exclusive percentage, demonstrating that most file hash feeds
are distinct from each other.

Due to the different sources of malware between feeds, a low intersection is to
be expected in some cases. For example, PA Twitter Emotet and PA Zeus Tracker
should have no intersection, since they are tracking different malware strains.
The other, more general feeds could expect some overlap, but mostly exhibit
little to no intersection. Considering the sheer volume of the FB Malware feed,
one might expect it would encapsulate many of the smaller feeds or at least
parts of them. This is not the case, however, as FB Malware has a negligible
intersection with all other feeds.

Due to the lack of intersection among the feeds, I omit the latency analysis
of the hash feeds, as there is simply not enough intersecting data to conclude
which feeds perform better with regards to latency.

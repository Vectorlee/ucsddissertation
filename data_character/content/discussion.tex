\section{Discussion}
\label{sec:discussion}

\subsection{Metrics Usage}
Threat intelligence has many different potential uses. For example,
analysts may consume threat data interactively during manual
incident investigations, or may use it to automate the detection of
suspicious activity and/or blacklisting.  When not itself
determinative, such information may also be used to \emph{enrich}
other data sources, informing investigations or aiding in automatic
algorithmic interventions. I have introduced a set of basic threat
intelligence metrics---volume, intersection, unique contribution,
latency, coverage and accuracy---that can inform and quantify each of
those uses.  Depending on a number of factors, such as the intended
use case and the cost of false positives and negatives, some of
these metrics will become more or less important when evaluating a
\ti\ source. For example, a feed with poor accuracy but
high coverage might be ideal when an analyst is using a \ti\ source
interactively during manually incident investigations (since in this
case, the analyst, as a domain expert, can provide additional filtering
of false positives). Similarly, latency might not be a critical metric
in a retrospective use case (e.g., post-discovery breach
investigation).  However, if an organization is looking for a
\ti\ source where the IPs are intended to be added to a firewall's
blacklist then accuracy and latency should likely be weighted over
coverage, assuming that blocking benign activity is more costly.

Another common real-world scenario is that a company has a limited
budget to purchase \ti\ sources and has a specific set of threats (\ie
botnet, brute-force) they are focused on mitigating. In such cases,
the metrics I have described can be used directly in evaluating \ti\
options, biasing twoards sources that maximize coverage of the most
relevant threats while limiting intersection.


\subsection{Data Labeling}
Threat intelligence IP data carries different meanings. To properly use this
data, it is critical to know what the indicators actually mean: whether
they are Internet scanners, members of a botnet or malicious actors who
had attacked other places before. I have attempted to group feeds by their
intended meaning in my analysis.

However, this category information, which primarily comes from \ti\ sources
themselves, is not always available. Feeds such as {\feedalienvault} and
Facebook Threat Exchange sources contain a significant number
of indicators labeled ``Malicious'' or ``Suspicious.'' The meanings of
these indicators are unclear, making it difficult for consumers to decide
how to use the data and the possible consequences.

For feeds that provide category information, it is sometimes too broad
to be meaningful. For example, multiple feeds in my collection
simply label their indicators as ``Scanner.'' Network scanning can
represent port scanning (by sending SYN packets), or a vulnerability scan (by
probing host for known vulnerabilities). The ambiguity here, as a result of
ad-hoc data labeling, again poses challenges for security experts when using
\ti\ data.

Recently, standard \ti\ formats have been proposed and developed, notably
IODEF~\cite{IODEF}, CybOX~\cite{CybOX} and STIX~\cite{STIX}, that try to
standardize the threat intelligence presentation and sharing. But these
standards focus largely on the data format. There is room
to improve these standards by designing a standard \emph{semantics} for threat
intelligence data.

\subsection{Limitations}
There are several questions that this study does not address. I attempted to collect data from a diverse set of
sources, including public feeds, commercial feeds and industrial exchange feeds,
but it is inherently not comprehensive. There are some prohibitively expensive or
publication-restricted data sources that are not available to us. More specialized measurement
work should be done in the future to further analyze the performance of these
expensive and exclusive data sources.

A second limitation is my visibility into how different companies use threat
intelligence operationally. For a company, perhaps the most useful kind of metric
measures how a threat intelligence source affects its main performance indicators
as well as its exposure to risk. Such metrics would require a deep integration
into security workflows at enterprises to measure the operation effect of decisions
made using threat intelligence. This would allow CIOs and CSOs to better understand
exactly what a particular threat intelligence product contributes to a company. As a
researcher, I do not use \ti\ operationally. A better understanding of operational
needs would help refine my metrics to maximize their utility for operations-driven consumers.

The third limitation is the lack of ground truth, a limitation shared
by all the similar measurement work. It is simply very difficult to obtain the full picture
of a certain category of threat, making it very challenging to precisely determine accuracy and coverage of feeds. In this
study, I used data from an Internet telescope and VirusTotal as a close approximation.
There are also a handful of cases where a security incident has been comprehensively studied by
researchers, such as the Mirai study~\cite{antonakakis2017understanding}, and such efforts
can be used to evaluate certain types of \ti\ data. But such studies are few in number.
One alternative is to try to establish the ground truth for a specific network.
For example, a company can record all the network traffic going in and out of its own network,
and identify security incidents either through its IDS system or manual forensic analysis.
Then it can evaluate the accuracy and coverage of a \ti\ feed under the context of its
own network. This can provide a customized view of \ti\ feeds.



%%
%% Below are comments
%%
\begin{comment}
\subsection{Independence between Metrics}
I evaluated a large range of \ti\ sources using six metrics, it is
interesting to see if there is any correlation between these metrics.
I already compared the relation of between volume and latency of IP feeds
in Section~\ref{sec:ip-timing} and showed that there is not a strong correlation. Similar cases
can also be observed, for example, between volume and accuracy that neither high or low
volume implies high accuracy in my feeds collection. Also, both IP and MD5 feeds show
that the exclusive rate is irrelevant with the feeds' size.

To formally address this question, I check the correlation coefficient between
\emph{Volume}, \emph{Exclusive Contribution}, \emph{Latency} and \emph{Accuracy}
in IP and MD5 feeds(when available). The result shows a low(below 0.5) correlation across all comparison.
This means when choosing between different \ti\ sources, I need to evaluate them comprehensively
on multiple perspectives, as different perspective does not imply each other.

Coverage correlate with volume in the scan feeds, which is not a surprise since volume naturally
implies coverage when the accuracy are close.


\subsection{Limited Coverage}
Most \ti\ indicators are exclusive to each feed, for both IP and MD5 sources.
Moreover, when comparing with broad sensor data in known categories like Internet scanning,
fewer than 2\% of ``known bad'' addresses appear in all the data sources we
analyzed. These suggest several non-exclusive possibilities. First is that the
underlying space of indicators (both IP addresses and malicious file hashes) is
large and each individual data sources can at best sample a small fraction
thereof. Second, different measurement methodologies---even for the same threat
category---will select for different sub distributions of the
underlying ``ground truth'' data.  Third, this last effect is likely
exacerbated by the fact that not all threats are experienced uniformly
across the Internet and thus, some kinds of methodologies, will skew
to either favor or disfavor targeted attacks.

This implies that \ti\ consumers should focus more on how the feeds related to the
specific threat that the consumers are mostly exposed to, instead of blindly trying to maximize
coverage or to find the ``best'' feed. Future work would be to correlate \ti\ feeds' data
with the companies or organizations' network traffic and evaluate feeds under specific environments.
\end{comment}

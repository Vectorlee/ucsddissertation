\section{Longitudinal Comparison}
\label{sec:new_vs_old}

In addition to the measurement period considered so far (December 1, 2017 to July 20, 2018), I also analyzed data from the same IP feeds from January 1, 2016 to August 31, 2016. These two measurement periods, 23 months apart, allow us to measure how these IP feeds have changed in two years. Table~\ref{tab:old-volume-overview-1} and
Table~\ref{tab:old-volume-overview-2} summarizes the differences between these two measurement periods. \colname{Avg. Rate} shows the percentage of daily rate changed over the old feeds.
The two columns under \colname{Unrt} show the unroutable rates of feeds in 2016 and 2018 separately. The two columns under \colname{CDN} present the number of IPs fall in CDN IP ranges in old and new data. In the table, \colname{2018} represents the current measurement period and \colname{2016} the period  January 1, 2016 to August 31, 2016.
\newcolumntype{H}{>{\setbox0=\hbox\bgroup}c<{\egroup}@{}}

\begin{table}[t!]
\centering
\caption{Data changes in IP feeds compared against the ones in 2016, \colname{Avg. Rate} shows the percentage of daily rate changed over the old feeds.
The two columns under \colname{Unrt} show the unroutable rates of feeds in 2016 and 2018 separately. The two columns under \colname{CDN} present the number of IPs fall in CDN IP ranges in old and new data.}
\label{tab:old-volume-overview}
\scriptsize
 \begin{tabular}{l r H H H H H r r r r }
 \toprule
& & & & & & & \multicolumn{2}{c}{\colname{Unroutable}} & \multicolumn{2}{c}{\colname{CDN}} \\
\cmidrule(lr){8-9}\cmidrule(l){10-11}
 \colname{Feed} & \colname{Avg. Rate} & \colname{Exclusive} & \colname{Unrt} & \colname{Unrt} & \colname{CDNs} & \colname{New} &   2016 &  2018 &  2016  &  2018 \\ %& \colname{Unrt} \\
  \midrule
  \textbf{Scan Feeds} \\
  %\cline{1-1}
  PA AlienVault IPs 	    & $+$1,347\%    & 63.3\%  	& 18218.0  & +0.0        & +0     & 100\%   & 0.0\%    & 0.0\%      & 0    & 0 \\
  PA Packetmail ram* 	    & $+$733\%   & 75.3\% 	 	& 6749.8   & +0.0        & +0     & 100\%   & <0.01\%    & <0.01\%  	& 0    & 0 \\
  Packetmail IPs 	        & $+$135\%   & 83.4\% 	 	& 11145.8  & +0.0        & +0     & 99.4\%  & 0.0\%    & 0.0\%    	& 0     & 0\\
  Paid IP Reputation 	    & $-$57\%   & 97.7\% 	 	& 59430.9  & -7.08       & -889   & 91.3\%  & 8.73\%    & 1.65\%  & 910     & 21\\
  PA Lab Scan 	            & $-$1\%    & 85.1\% 	 	& 5921.8   & +<0.01      & +0     & 99.8\%  & 0.0\%     & <0.01\%   & 0   & 0 \\
  PA Snort BlockList 	    & $-$97\%   & 99.4\% 	 	& 123093.3 & +0.41       & -1     & 86.6\%  & <0.01\%     & 0.42\% 	& 1     & 0\\
  FB Aggregator$_1$ 	    & $+$332\%   & 83.8\% 	 	& 617.3    & +0.0        & -6     & 100\%   & 0.0\%     & 0.0\%     & 6   & 0 \\
  PA Analyst 	            & $-$44\%   & 52.5\% 	 	& 137.2    & +0.41       & +0     & >99.9\% & 0.0\%  	& 0.41\%    & 0   & 0\\

  %\midrule
  \textbf{Botnet Feeds} \\
  %\cline{1-1}
  PA CI Army 	            & $+$114\%   & 96.8\% 	& 4996.6  & +0.0     & +0    & 100\%  & <0.01\%  & <0.01\% & 0    & 0\\
  Paid IP Reputation 	    & $-$39\%   & 99.8\% 	& 17249.7 & +1.02    & +59   & 80.5\% & 0.63\%   & 1.66\%  & 15    & 74\\
  PA Botscout IPs 	        & $+$1\%    & 74.5\% 	& 5762.6  & +0.08    & -1    & 100\% & 0.01\%    & 0.09\%  & 1     & 0\\
  PA VoIP Blacklist 	    & $+$252\%   & 89.7\% 	& 732.4   & +0.32  	 & +0    & 100\% & 0.0\%     & 0.32\%  & 0    & 0\\
  PA Compromised IPs 	    & $-$36\%   & 94.9\% 	& 1515.8  & -0.10 	 & +0    & 100\% & 0.10\%    & 0.0\%   & 0   & 0\\
  PA Blocklist Bots 	    & $-$95\%	  & 82.0\% 	& 8282.6  & +0.0     & +0    & 100\% & 0.0\%     & 0.0\%   & 0    & 0 \\
  PA Project Honeypot 	    & $+$63\%	  & 48.3\% 	& 260.2   & +0.0     & +0    & 100\% & 0.0\%     & 0.0\%   & 0    & 0 \\

  %\midrule
  \textbf{Brute-force Feeds} \\
  %\cline{1-1}
   Badips SSH 	             & $+$30\%   & 94.2\% 	& 45479.9   & +0.12    & +1     & 97.6\% & 0.07\%      & 0.19\% & 0   & 1    \\
   Badips Badbots 	         & $+$1,732\%    & 87.6\%     & 1085.5    & +1.04    & +1064  & 92.6\% & 0.0\%      & 1.04\%  & 187 & 1,251   \\
   Paid IP Reputation 	     & $-$62\%   & 91.0\% 	& 19483.5   & -6.52    & -325   & 98.4\% & 6.55\%    & 0.03\%   & 335 & 10  \\
   PA Brute-Force 	         & $-$72\%   & 97.3\% 	& 63016.8   & +0.0     & +0     & 100\%  & 0.0\%      & 0.0\%   & 0   & 0  \\
   Badips Username*          & $+$3,040\%    & 16.6\% 	& 175.7     & +0.53    & +0     & 95.7\% & 0.0\%     & 0.53\% 	& 0  & 0\\
   Haley SSH 	             & $+$428\%   & 13.6\% 	& 226.6     & -0.01    & +0     & 94.8\% & 0.04\% 	& 0.03\%    & 0  & 0\\
   FB Aggregator$_2$ 	     & $+$387\%   & 42.4\% 	& 358.3     & -0.12    & +0     & 100\%  & 0.12\% 	& 0.0\%     & 0  & 0\\
   Nothink SSH 	             & $+$886\%   & 53.8\% 	& 778.5     & 0.95     & +0     & 86.4\% & 0.56\%	& 1.51\%    & 0  & 0\\
   Dangerrulez Brute 	     & $+$0\%	   & 9.15\% 	& 1109.8    & +0.0     & -1     & 99.8\% & 0.0\%    & 0.0\% 	& 1  & 0 \\

  %\midrule
  \textbf{Malware Feeds} \\
  %\cline{1-1}
  Paid IP Reputation 	       & $-$36\%	 & 96.2\% 	& 35273.7 	& -0.05     & -11,776   & 94.3\%& 0.18\%   & 0.13\% & 15265     & 3,489\\
  FB Malicious IPs 	           & $-$77\%   & 99.6\% 	& 17649.6   & -4.4      & -264      & 99.9\% & 6.81\%  & 2.14\% & 264     & 0   \\
  Feodo IP Blacklist 	       & $+$0\%    & 24.6\% 	& 589.0     & +0.0      & +0	    & 52.4\% & 0.0\%   & 0.0\% 	& 0      & 0 \\
  Malc0de IP Blacklist 	       & $-$9\%    & 60.0\%   & 143.1     & +0.0   	& -121      & 99.2\% & 0.0\%   & 0.0\%  & 132    & 11\\
  PA Bambenek C2 IPs 	       & $+$79\%   & 74.3\% 	& 72.9      & +9.13     & +0	    & 100\%  & 0.0\%   & 9.13\% & 0     & 0 \\
  PA SSL Malware IPs 	       & $-$34\%   & 30.1\% 	& 73.0      & +0.0      & +0	    & 100\%  & 0.0\%   & 0.0\%  & 0   & 0 \\
  PA Analyst 	               & $-$93\%    & 76.4\% 	& 917.7     & +0.0      & +0        & 99.8\% & 0.34\%  & 0.0\%  & 0  & 0\\
  PA Abuse.ch* 	               & $-$99\%   & 2.19\% 	& 4755.2    & +2.63     & +0        & 44.9\% & 0.49\%  & 3.12\% & 0  & 0\\
  PA Mal-Traffic-Anal  	       & $-$53\%   & 39.0\% 	& 25.7      & +0.51     & +0        & 100\%  & 0.0\%   & 0.51\%  & 0    & 0 \\
  Zeus IP Blacklist 	       & $-$66\%   & 24.3\%   & 164.8     & +0.0      & -6        & 49.2\% & 0.0\%   & 0.0\%   & 6  & 0\\


  %\midrule
  \textbf{Exploit Feeds} \\

Badips HTTP    & $+$326\%    & 98.0\% 	& 11693.1 	& +0.37  & +2,154  & 97.8\%   & 0.30\%  & 0.67\%  & 436 & 2,590 \\
Badips FTP 	   & $+$556\%    & 98.1\% 	& 6132.7    & +1.32  & +2      & 82.7\%   & 0.01\%  & 1.33\%  & 0   & 2\\
Badips DNS 	   & $+$9,525\%     & 96.9\% 	& 51.8      & +0.33  & +237    & 98.8\%   & 0.17\%  & 0.50\%  & 7   & 244  \\
Badips RFI 	   & $+$226\%    & 29.1\%  & 40.9      & +2.22  & -1      & 67.6\%   & 0.0\%   & 2.22\%  & 0   & 0  \\
%Badips SQL 	   & 66.6\% 	& 0.0 	& 0.0 \\

 \textbf{Spam Feeds} \\
  %\cline{1-1}
Paid IP Reputation 	 & $+$133\%	     & 99.9\%      & 4626.5    & +19.4   & +0      & 89.7\%  & 59.3\%    & 78.7\% & 0   & 0\\
Badips Spam 	     & $+$12,767\%    & 40.0\%      & 320.5     & +0.02   & +0      & 99.0\% 	& 0.0\%     & 0.02\% & 0   & 0 \\
Badips Postfix 	     & $-$53\%       & 99.1\%      & 52594.8   & +1.28   & +1      & 96.6\%  & <0.01\%   & 1.29\% & 0   & 1\\
PA Botscout IPs 	 & $+$18\%	     & 97.5\% 	  & 4337.3    & +0.06   & +0      & >99.9\%	& 0.0\%     & 0.06\% & 0   & 0 \\
AlienVault IP Rep 	 & $+$8\%	    & 87.2\% 	  & 1497.7    & -0.50   & +561    & 98.8\% 	& 0.57\%	& 0.07\% & 479  & 1,040 \\


\bottomrule
\end{tabular}
\end{table}



%\caption{Data changes in IP feeds compared against the ones in 2016, \colname{Avg. Rate} shows the change percentage of daily rate.
%For small changes we use ``\%'', for large changes we use ``x'', indicating x times.
%\colname{Unrt} shows the changes between unroutable rate with before. \colname{CDNs} are the changes on the number of IPs fail in into the CDN IP ranges.
%\colname{New} shows the percentage of indicators that are new to the recent feed compared with the old feed.}


\textbf{Volume.}
As shown in Table~\ref{tab:old-volume-overview-1} and ~\ref{tab:old-volume-overview-2}, feed volume has definitely changed after two years. Among 43 IP feeds that overlap both time periods,
21 have a higher daily rate compared with 2 years ago, 15 feeds
have a lower rate, and 7 feeds do not change substantially (the difference is below 20\%).
Volume can change dramatically over time, such as {\feedTSAlienVault}
in the scan category which is 13 times larger than before. On the other hand, a feed like {\feedTSBots} is now over 90\% smaller.

\textbf{Intersection and Exclusive Contribution.}
Despite the volume differences, the intersection statistics between feeds are largely the same across two years,
with feeds in scan and brute-force having high pairwise intersections and
feeds in other categories being mostly unique. Certain specific pairwise relations also did not change.
For example, {\feedbadipssh} still shared over 90\% of data in {\feeddangerrule} back in 2016, and {\feedetiprep} in malware
was still the only feed that has a non-trivial intersection with multiple small feeds.
Again, most data was exclusive to each feed two years ago: Across all
six categories more than 90\% of the indicators are not shared between feeds.

\textbf{Latency.}
The latency relationship between feeds was also similar:
timely feeds today were also timely two years ago, and the same with tardy feeds.
%The latency difference between feeds are bigger then compared with data, but small feeds
%still report a significant portion of their shared data.

\textbf{Accuracy.}
Feeds have more unroutable IPs now than before as shown in Table~\ref{tab:old-volume-overview-1} and~\ref{tab:old-volume-overview-2}:
In 2016, 22 of the 43 IP feeds had at least 1 unroutable IP; four feeds had unroutable rates over 1\%.
When checking the intersection with popular CDNs,
the feeds that contain IPs in CDN ranges two years ago are also the ones that have these IPs today.

\textbf{Shared indicators 2016--2018.}
I compared the data I collected from each feed in the two time periods, and found that 30
out of 43 feeds in 2018 intersect with their data from two years ago, and 9 feeds have
an intersection rate over 10\%. Three feeds in malware category, namely {\feedfeodo},
{\feedTSAbusech} and {\feedzeus}, have over 40\% of their data shared with the past feed,
meaning a large percent of C\&C indicators two years ago are still
identified by the feeds as threats today. Feeds in the botnet category, however, are very distinct from the
past, with all feeds having no intersection with the past except {\feedetiprep}.
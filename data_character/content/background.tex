\section{Related Work}
\label{sec:background}

Several studies have examined the effectiveness of blacklist-based threat
intelligence~\cite{kuhrer2014paint, ramachandran2006revealing, ramachandran2007filtering, sheng2009empirical, sinha2008shades}.
Ramachandran~\etal~\cite{ramachandran2007filtering} showed that spam blacklists
are both incomplete (missing 35\% of the source IPs of spam emails captured in
two spam traps), and slow in responding (20\% of the spammers remain unlisted
after 30 days). Sinha~\etal~\cite{sinha2008shades} further confirmed this
result by showing that four major spam blacklists have very high false negative
rates, and analyzed the possible causes of the low coverage.
Sheng~\etal~\cite{sheng2009empirical} studied the effectiveness of
phishing blacklists, showing the lists are slow in reacting to
highly transient phishing campaigns. These studies focused on specific
types of threat intelligence sources, and only evaluated their operational
performance rather than producing empirical evaluation metrics for
threat intelligence data sources.

Other studies have analyzed the general attributes of threat
intelligence data.  Pitsillidis~\etal~\cite{tasters:imc12} studied the
characteristics of spam domain feeds, showing different perspectives
of spam feeds, and demonstrated that different feeds are suitable for
answering different questions.  Thomas~\etal~\cite{thomas2016abuse}
constructed their own threat intelligence by aggregating the abuse
traffic received from six Google services, showing a lack of
intersection and correlation among these different sources.  While
focusing on broader threat intelligence uses, these studies did not
focus on generalizable threat metrics that can be extended beyond the work.

Little work exists that defines a general measurement methodology to
examine threat intelligence across a broad set of types and categories.
Metcalf~\etal~\cite{metcalf2015blacklist} collected and measured IP
and domain blacklists from multiple sources, but only focused on volume
and intersection analysis. In contrast, we formally define a set of threat intelligence
metrics and conduct a broad and comprehensive study over a rich variety of
threat intelligence data. We conducted our measurement from the
perspective of consumers of \ti\ data
to offer guidance on choosing between different sources.
Our study also demonstrated the limitation of threat intelligence more thoroughly,
providing comprehensive characteristics of cyber threat intelligence that no work had
addressed previously.

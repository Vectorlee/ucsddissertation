\subsection{Threat Intelligence Metrics}
\label{sec:metrics}

%\note{Should explain why each of these metrics is useful.}
The aim of this work is to develop \emph{threat intelligence metrics} that
allow a \ti\ consumer to compare threat intelligence sources and reason about
their fitness for a particular purpose. To this end, we propose six concrete
metrics: \emph{Volume}, \emph{Differential contribution}, \emph{Exclusive contribution},
\emph{Latency}, \emph{Accuracy} and \emph{Coverage}.

%\noteby{KL}{This needs to be updated to \emph{size} and \emph{rate}.}

\metrics \emphpar{Volume} We define the \emph{volume} of a feed to be the total
number of indicators appearing in a feed over the measurement interval. Volume
is the simplest \ti\ metric and has an established history in prior
work~\cite{jung2004empirical,kuhrer2014paint,
trajectory:oakland11,tasters:imc12,sheng2009empirical,sinha2008shades,thomas2016abuse}.
It is also useful to study the daily \emph{rate} of a feed, which quantifies
the amount of data appearing in a feed on a daily basis.

\emph{Rationale}: To a first approximation, volume captures how much
information a feed provides to the consumer. For a feed without false positives
(see \emph{accuracy} below), and if every indicator has equal value to the
consumer, we would prefer a feed of greater volume to a feed of lesser volume.
Of course, indicators do not all have the same value to consumers: knowing the
IP address of a host probing the entire Internet for decades-old
vulnerabilities is less useful than the address of a scanner targeting
organizations in your sector looking to exploit zero-day vulnerabilities.

\metrics \emphpar{Differential contribution} The \emph{differential
contribution} of one feed with respect to another is the number of indicators
in the first that are not in the second during the same measurement period. We
define differential contribution relative to the size of the first feed, so
that the differential contribution of feed $A$ with respect to feed $B$ is
$\mathrm{Diff}_{A,B} = |A \setminus B|/|A|$. Thus, $\mathrm{Diff}_{A,B}=1$
indicates that the two feeds have no elements in common, and
$\mathrm{Diff}_{A,B} = 0$ indicates that every indicator in $A$ also appears in
$B$. It is sometimes useful to consider the complement
of differential contribution, namely the normalized \emph{intersection} of $A$ in $B$,
given by $\mathrm{Int}_{A,B} = |A \cap B|/|A| = 1-\mathrm{Diff}_{A,B}$.

%(Figure~\ref{fig:overall_heatmap}, in Section~\ref{sec:ip-overlap}, for
%example, shows intersection rather than differential contribution, following the
%Taster's Choice convention.)

\emph{Rationale}: For a consumer, it is often useful to know how many \emph{additional}
indicators a feed offers relative to one or more feeds that the consumer has already.
Thus, if a consumer already has feed $A$ and is considering paying for feed $B$,
then $\mathrm{Diff}_{A,B}$ indicates how many new indicators feed $A$ will provide.

\metrics \emphpar{Exclusive contribution} The \emph{exclusive contribution} of
a feed with respect to a set of other feeds is the proportion of indicators
unique to a feed, that is, the proportion of indicators that occur in the feed
but no others. Formally, the exclusive contribution of feed $A$
is defined as $\mathrm{Uniq}_{A,B} = |A \setminus \bigcup_{B\neq A}B|/|A|$.
Thus, $\mathrm{Uniq}_{A,B} = 0$ means that every element of feed $A$ appears in
some other feeds, while $\mathrm{Uniq}_{A,B}=1$ means no element of $A$ appears
in any other feed.

%Exclusive contribution was first described for domain feeds
%in Taster's Choice~\cite{tasters:imc12}.

\emph{Rationale}: Like differential contribution, exclusive contribution tells
a \ti\ consumer how much of a feed is different. However, exclusive contribution
compares a feed to all other feeds available for comparison, while differential
contribution compares a feed to just another feed. From a \ti\ consumer's
perspective, exclusive contribution is a general measure of a feed's unique
value.
%A feed that aggregates several other feeds may have no exclusive
%contribution, but may provide great value to a consumer.

\metrics \emphpar{Latency} For an indicator that occurs in two or more feeds, its
\emph{latency} in a feed is the elapsed time between its first appearance in any
feed and its appearance in the feed in question. In the feed where an indicator
first appeared, its latency is zero. For all other feeds, the latency indicates
how much later the same indicators appears in those feeds. Taster's
Choice~\cite{tasters:imc12} referred to latency as \emph{relative first appearance
time}. (We find the term \emph{latency} to be more succinct without loss of
clarity.) Since latency is defined for one indicator, for a feed it makes sense
to consider statistics of the distribution of indicator latencies, such as the
median indicator latency.
%In the rest of the paper, we use the term \emph{feed latency} as shorthand for a latency statistic.

\emph{Rationale}: Latency characterizes how quickly a feed includes new
threats: the sooner a feed includes a threat, the more effective it is at
helping consumers protect their systems. Indeed, several studies report on the
impact of feed latency on its effectiveness at
thwarting spam~\cite{blacklisting:weis14,ramachandran2007filtering}.

The metrics above are defined without regard for the \emph{meaning} of the
indicators in a feed. We can calculate the volume of a single feed or the
differential contribution of one feed with respect to another regardless of
what the feed purports to contain. While these metrics are easy to compute,
they do little to tell us about the fitness of a feed for a particular purpose.
For this, we need to consider the meaning or purpose of the feed data, as
advertised by the feed provider. We define the following two metrics.

%Unfortunately, this is not always easy to do.
%In our experience, few feeds are described in such detail as to be able to say,
%unequivocally, whether a particular indicator should be properly included in a
%feed or not. Nevertheless, for some feeds we can speak with some confidence
%about the kinds of indicators they should contain. For such feeds,


\metrics \emphpar{Accuracy} The \emph{accuracy} of a feed is the proportion of
indicators in a feed that are correctly included in the feed. Feed accuracy is
analogous to \emph{precision} in Information Retrieval. This metric presumes
that the description of the feed is well-defined and describes a set of elements
that should be in the feed given perfect knowledge. In practice, we have neither
perfect knowledge nor a perfect description of what a feed should contain. In
some cases, however, we can construct a set $A^-$ of elements that should
definitely not be in a feed $A$. Then $\mathrm{Acc}_A \le |A\setminus A^-|/|A|$.

\emph{Rationale}: The accuracy metric tells a \ti\ consumer how many false
positives to expect when using a feed, and, therefore, dictates how a feed
can be used. For example, if a consumer automatically blocks all traffic to
IP addresses appearing in a feed, then false positives may cause disruption
in an enterprise by blocking traffic to legitimate sites. On the other hand,
consumers may tolerate some false positives if a feed is only used to
gain additional insight during an investigation.

\metrics \emphpar{Coverage} The \emph{coverage} of a feed is the proportion
of the intended indicators contained in a feed. Feed coverage is analogous
to \emph{recall} in Information Retrieval. Like accuracy, coverage presumes
that the description of the feed is sufficient to determine which elements
should be in a feed, given perfect knowledge. In some cases, it is possible
to construct a set $A^+$ of elements that should be in a feed. We can then
upper-bound the coverage $\mathrm{Cov}_A \le |A|/|A^+|$.


\emph{Rationale}: For a feed consumer who aims to obtain complete protection
from a specific kind of threat, coverage is a measure of how much protection
a feed will provide. For example, an organization that wants to protect itself
from a particular botnet will want to maximize its coverage of that botnet's
command-and-control servers or infection vectors.

In the following two sections, we use these metrics to evaluate two types of
\ti: IP address feeds and file hash feeds.

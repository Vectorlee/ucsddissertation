\subsection{Accuracy}
\label{sec:ip-accuracy}
\newcolumntype{H}{>{\setbox0=\hbox\bgroup}c<{\egroup}@{}}

\begin{table}[t!]
\centering
\caption{IP \ti\ feeds accuracy overview. \colname{Unrt} is fraction of unroutable addresses in each feed (Section~\ref{sec:ip-accuracy}).
\colname{Alexa Top} is the number of IPs intersected with top Alexa domain IP addresses, and
\colname{CDNs} is the number of IPs intersected with top CDN provider IP addresses.}
\label{tab:accuracy-overview-1}
\scriptsize
 \begin{tabular}{l r r r r}
 \toprule
 \colname{Feed} & \colname{Added} & \colname{Unrt} & \colname{Alexa} &  \colname{CDNs} \\ %& \colname{Unrt} \\
  \midrule
  \textbf{Scan Feeds} \\
  %\cline{1-1}
{\feedTSAlienVault}  & 313,175 	& 0.0\% 	& 1  & 0 \\
{\feeddshield}       & 339,805 	& 0.03\% 	& 68 & 62\\
{\feedTSramnode}     & 200,568 	& <0.01\% 	& 0  & 0 \\
{\feedpacketmail}    & 211,081 	& 0.0\% 	& 0  & 0 \\
{\feedetiprep}       & 200,915 	& 1.65\% 	& 6  & 21\\
{\feedTSLabScan}     & 169,037 	& <0.01\% 	& 0  & 0 \\
{\feedTSSnort}       & 12,957 	& 0.42\% 	& 1  & 0 \\
{\feedFBBasecamp}    & 5,601 	& 0.0\% 	& 0  & 0 \\
{\feedTSAnalyst}     & 1,451 	& 0.41\% 	& 0  & 0 \\


  %\midrule
  \textbf{Botnet Feeds} \\
  %\cline{1-1}
{\feedTSAnalyst}     & 180,034 	& <0.01\% 	& 0  & 0 \\
{\feedTSCI}          & 76,125 	& <0.01\% 	& 0  & 0 \\
{\feedetiprep}       & 73,710 	& 1.66\% 	& 6  & 74\\
{\feedTSBotscout}    & 18,638 	& 0.09\% 	& 1  & 0 \\
{\feedTSVoIP}        & 9,290   	& 0.32\% 	& 0  & 0 \\
{\feedTSCompr}       & 4,883 	    & 0.0\% 	& 0  & 0 \\
{\feedTSBots}        & 3,594 	    & 0.0\% 	& 0  & 0 \\
{\feedTSHoneypot}    & 1,947 	    & 0.0\% 	& 0  & 0 \\


  %\midrule
  \textbf{Brute-force Feeds} \\
  %\cline{1-1}
{\feedbadipssh}     & 456,605 	& 0.19\% 	& 217  & 1\\
{\feedbadipbot}     & 91,553 	& 1.04\% 	& 46   & 1,251\\
{\feedetiprep}      & 87,524 	& 0.03\% 	& 0  & 10\\
{\feedTSBrute}      & 31,555 	& 0.0\% 	& 0  & 0\\
{\feedusername}     & 37,198 	& 0.53\% 	& 4  & 0\\
{\feeddisco}        & 8,784 	& 0.03\% 	& 0  & 0\\
{\feedFBZendesk}    & 17,779 	& 0.0\% 	& 0  & 0\\
{\feednothink}      & 20,325 	& 1.51\% 	& 2  & 0\\
{\feeddangerrule}   & 8,247 	& 0.0\% 	& 0  & 0\\


  %\midrule
  \textbf{Malware Feeds} \\
  %\cline{1-1}

{\feedetiprep}       & 217,073 	& 0.13\% 	& 291  & 3,489\\
{\feedFBAdmin}       & 29,840 	& 2.14\% 	& 2    & 0\\
{\feedfeodo}         & 296 	& 0.0\% 	& 0   & 0\\
{\feedTSLabMalware}  & 806 	& 2.85\% 	& 0   & 0\\
{\feedmalcode}       & 668 	& 0.0\% 	& 8   & 11\\
{\feedTSBambenek}    & 777 	& 9.13\% 	& 0   & 0\\
{\feedTSSSL}         & 674 	& 0.0\% 	& 0   & 0\\
{\feedTSAnalyst}     & 486 	& 0.0\% 	& 0   & 0\\
{\feedTSAbusech}     & 256 	& 3.12\% 	& 0   & 0\\
{\feedTSMalTraffic}  & 193 	& 0.51\% 	& 0   & 0\\
{\feedzeus}          & 67 	& 0.0\% 	& 1   & 0\\

  %\midrule
  \textbf{Exploit Feeds} \\
{\feedbadiphttp}    & 305,020 	& 0.67\% 	& 16  & 2,590\\
{\feedbadipftp}     & 285,329 	& 1.33\% 	& 14  & 2\\
{\feedbadipdns}     & 46,813 	& 0.50\% 	& 119 & 244 \\
{\feedbadiprfi}     & 3,642 	& 2.22\% 	& 0   & 0\\
{\feedbadipsql}     & 737 	& 1.89\% 	& 0   & 1\\

  \textbf{Spam Feeds} \\
  %\cline{1-1}
{\feedetiprep}      & 543,546 	& 78.7\% 	& 1	& 0 \\
{\feedbadipspam}    & 302,105 	& 0.02\% 	& 19	& 0 \\
{\feedbadippostfix} & 193,674 	& 1.29\% 	& 18	& 1 \\
{\feedTSBotscout}   & 11,358 	& 0.06\% 	& 0  	& 0 \\
{\feedalienvault}   & 10,414 	& 0.07\% 	& 63	& 1,040 \\

\bottomrule
\end{tabular}
\end{table}

%\footnotetext{}


Accuracy measures the rate of false positives in a feed. A false
positive is an indicator that data is labeled with a category to which
it does not belong.  For example, an IP address found in a scan feed
that has not conducted any Internet scanning is one such false
positive.  As well, even if a given IP is in fact associated with
malicious activity, if it is not unambiguously actionable (e.g.,
Google's DNS at 8.8.8.8 is used by malicious and benign software
alike) then for many use cases it must also be treated as a false
positive.  False positives are problematic for a variety of reasons,
but particularly because they can have adverse operational
consequences.  For example, one might reasonably desire to block all
new network connections to and from IP addresses reported as hosting
malicious activity (indeed, this use is one of the promises of threat
intelligence). False positives in such feeds, though, could lead to
blocking legitimate connections as well.  Thus, the degree of accuracy
for a feed may preclude certain use cases.

Unfortunately, determining which IPs belong in a feed and which do not
can be extremely challenging. In fact, at any reasonable scale, we are
unaware of any method for unambiguously and comprehensively
establishing ``ground truth'' on this matter.  Instead, in this
section we report on a proxy for accuracy that provides a
conservative assessment of this question.  To wit, we assemble a
\emph{whitelist} of IP addresses that either should not reasonably be
included in a feed, or that, if included, would cause significant
disruption. We argue that the presence of such IPs in a feed are
clearly false positives and thus define an upper bound on a feed's
accuracy.  We populate our list from three sources: unroutable IPs,
IPs associated with top Alexa domains, and IPs of major content
distribution networks (CDNs). The detail result for our accuracy analysis 
is presented in Table~\ref{tab:accuracy-overview-1} and 
Table~\ref{tab:accuracy-overview-2}.
\colname{Unrt} is fraction of unroutable addresses in each feed
(Section~\ref{sec:ip-accuracy}).
\colname{Alexa Top} is the number of IPs intersected with top Alexa domain 
IP addresses, and \colname{CDNs} is the number of IPs intersected with 
top CDN provider IP addresses. I will explain more about each column in below.

\noindent\textbf{Unroutable IPs.} Unroutable IPs are IP addresses that
were not BGP-routable \emph{when they first appeared} in a feed, as
established by contemporaneous data in the RouteViews
service~\cite{Routeview}. While such IPs could have appeared in the
source address field of a packet (i.e., due to address spoofing), it
would not be possible to complete a TCP handshake. Feeds that imply
that such an interaction took place should not include such IPs. For
example, feeds in the Brute-force category imply that the IPs they
contain were involved in brute-force login attempts, but this could
not have taken place if the IPs are not routable. While including
unroutable addresses in a feed is not, in itself, a problem, their
inclusion suggests a quality control issue with the feed, casting
shade on the validity of other indicators in the feed.

To allow for some delays in the feed, we check if an IP was routable
at any time in the seven days prior to its first appearance in a feed,
and if it had, we do not count it as
unroutable. Table~\ref{tab:accuracy-overview-1}, column \textit{Unrt},
shows the fraction of IP indicators that were not routable at any time
in the seven days prior to appearing in the feed. This analysis is
only conducted for the IPs that are added after our measurement
started. The number of such IPs is shown in column \textit{Added}, and
the unroutable fraction shown in \textit{Unrt} is with respect to this
number.

\noindent\textbf{Alexa.} Blocking access to popular Internet sites or
triggering alarms any time such sites are accessed would be disruptive
to an enterprise. For our analysis, we periodically collected the
Alexa top 25 thousand domains (3--4 times a month) over the course of
the measurement period~\cite{alexa}. To address the challenge that
such lists can have significant churn~\cite{scheitle2018long}, we
restrict our whitelist to hold the \emph{intersection} of all these
top 25K lists (i.e., domains that were in the top 25K every time we
polled Alexa over our 8-month measurement period), which left us with
12,009 domains.   We then queried DNS for the A records, NS
records and MX records of each domain, and collected the corresponding
IP addresses. In total, we collected 42,436 IP addresses associated
with these domains. We compute the intersection of these IPs
with \ti\ feeds and show the results in column \textit{Alexa} in
Table~\ref{tab:accuracy-overview-1}.


\noindent\textbf{CDNs.} CDN providers serve hundreds of thousands of
sites. Although these CDN services can (and are) abused to conduct
malicious activities~\cite{cdnabuse}, their IP addresses are not
actionable.  Because these are fundamentally shared services,
blocking such IP addresses will also disrupt access to benign
sites served by these IPs.  We collected the IP ranges used by 5
popular CDN providers: AWS CloudFron~\cite{cloudfront},
Cloudflare~\cite{cloudflare}, Fastly~\cite{fastly},
EdgeCast~\cite{edgecast} and MaxCDN~\cite{maxcdn}. We then check how
many IPs in \ti\ feeds fall into these ranges. Column \textit{CDNs} in
Table~\ref{tab:accuracy-overview-1} shows the result.

\finding\ Among the \numipfeeds\ feeds in the table, 33 feeds have at
least one unroutable IP, and for 13 of them, over 1\% of the addresses
they contain are unrouteable. Notably, the {\feedetiprep} feed in the
spam category has an unroutable rate over 78\%.  Although it is not
documented, a likely explanation is that this feed may include unroutable
IPs intentionally, as this is a known practice among certain spam
feeds. For example, the Spamhaus DROP List~\cite{Spamhaus} includes IP
address ranges known to be owned or operated by malicious actors,
whether currently advertised or not. Thus, for feeds that explicitly
do include unroutable IPs, their presence in the feeds should not
necessarily be interpreted as a problem with quality control.

We further checked feeds for the presence of any ``reserved IPs''
which, as documented in RFC 8190, are not globally routable (e.g., private
address ranges, test networks, loopback and multicast).  Indeed,
12 feeds reported at least one reserved IP, including four of the
{\feedetiprep} feeds (excepting the spam category), six of the Badips
feeds, and the {\feedFBAdmin} and {\feeddshield} feeds. Worse, the
{\feedetiprep} feeds together reported over 100 reserved IPs. Since
such addresses should never appear on a public network,
reporting such IPs indicates that a feed provider fails to
incorporate some basic sanity checks on its data.

There are 21 feeds that include IPs from top Alexa domains, as shown
in column \textit{Alexa} in Table~\ref{tab:accuracy-overview-1}. Among
these IPs there are 533 A records, 333 IPs of MX records and 63 IPs of
NS records.  The overlapped IPs include multiple instances from
notable domains. For example, the IP addresses of www.github.com are
included by {\feedmalcode}.  {\feedetiprep} in the malware category
contains the IP address for www.dropbox.com.  {\feedalienvault}
contains the MX record of groupon.com, and {\feedbadipssh} also
contains the IP addresses of popular websites such as www.bing.com.

Most of the feeds we evaluated do not contain IPs in CDN ranges, yet
there are a few (including multiple {\feedetiprep} feeds, Badips feeds
and {\feedalienvault}) that have significant intersection with CDN
IPs. {\feedalienvault} and Badips feeds primarily intersect with
%IPs from
Cloudflare CDN, while most of the overlap in the {\feedetiprep} malware
category overlaps with AWS CloudFront.

Overall, the rate of false positives in a feed is not strongly
correlated with its volume.  Moreover, certain classes of false
positives (e.g., the presence of Top Alex IPs or CDN IPs) seem to
be byproducts of how distinct feeds are collected (e.g., Badips
feeds tend to contain such IPs, irrespective of volume).
Unsurprisingly, we also could find not correlation between a feed's
latency and its accuracy.


%This error will largely distort our other analyses, so for all the other analyses in this paper, we removed the data on that days from two feeds.

%Some IP threat intelligence feeds, like Spamhaus DROP list\cite{Spamhaus}, contain IP address blocks, which include all the IP addresses in a subnet. These netblocks might cause a feed to include unroutable IP addresses. To further verify this possibility, we checked all the newly added IPs of each feed on each day, and found none of the feeds have reported consecutive IP addresses that span more than 32 IP address.

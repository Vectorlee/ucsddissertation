\section{Overview}
\label{sec:overview}

The threat intelligence data collected for this study was obtained by
subscribing to and pulling from numerous public and private 
intelligence sources. These sources ranged from simple blacklists of 
bad IPs/domains and file hashes, to rich threat intelligence exchanges 
with well labeled and structured data.
I refer each item (\eg IP address or file hash) an \emph{indicator}
(after \emph{indicator of compromise}, the industry term for such data items).

In this section, I enumerate the threat intelligence sources, describe
each source's structure and how I collected it, and then define our
measurement metrics for empirically measuring these sources.  When the
source of the data is public, or when I have an explicit agreement to
identify the provider, I have done so.  However, in other cases, the
data was provided on the condition of anonymity and I restrict
myself to only describe the nature of the provider, but not their
identity.  All of the private data providers were appraised of the
nature of my research, its goals and the methodology that I planned
to employ.


%\pp{Try to use feeds when describing all except FB and TS. I've reworked such
%that feeds -> sources}

\subsection{Data Set and Collection}

I use several sources of \ti\ data for my analysis:

\emphpar{Facebook ThreatExchange (FB)~\cite{FBThreatExchange}}
    This is a closed-community platform that allows hundreds of companies and
    organizations to share and interact with various types of labeled threat data.
    As part of an agreement with Facebook, I collected all its data that it shared broadly.
    In subsequent analyses, sources with
    prefix ``FB'' indicate a unique contributor on the Facebook ThreatExchange.

\emphpar{Paid Feed Aggregator (PA)} This is a commercial paid threat intelligence data aggregation
    platform. It contains data collected from over a hundred other
    threat intelligence sources, public or private, together with its own threat data.
    In subsequent analyses all data sources with prefix ``PA''
    are from unique data sources originating from this aggregator.

\emphpar{\feedetiprep\ Service} This commercial service provides an
    hourly-updated blacklist of known bad IP addresses across different attack categories.

\emphpar{Public Blacklists and Reputation Feeds} I collected indicators from public
    blacklists and reputation data sources, including well-known
    sources such as AlienVault~\cite{Alienvault}, Badips~\cite{Badips}, Abuse.ch~\cite{Abuse-ch}
    and Packetmail~\cite{Packetmail}.

Threat Intelligence indicators include different types of data, such as IP address, malicious file hash, Domain,
URL, etc. In this chapter, I focus the analysis on sources that provide IP addresses and file hashes,
as they are the most prevalent data types in my collection.

I collect data from all sources on an hourly basis. However, both the Facebook
ThreatExchange and the Paid Feed Aggregator change their members and contributions
over time, creating irregular collection periods for several of the sub-data sources.
Similarly, public threat feeds had varying degrees of reliability, resulting in
collection gaps. In this chapter, I use the time window from \veryemph{December 1, 2017}
to \veryemph{July 20, 2018} for most of the analyses, as I have the largest number
of active sources during this period. I eliminated duplicates sources (e.g., sources
we collected individually and also found in the {Paid Aggregator}) and
sub-sources (a source that is a branch of another source). I further break IP
sources into separate categories and treat them as individual feeds, as shown in
Section~\ref{sec:ip-analysis}. This filtering leaves us with \numipfeeds\ IP feeds and
\numhashfeeds\ malware file hash feeds.

The ways each \ti\ source collects data varies, and in some cases
the methodology is unknown. For example, {\feedpacketmail} and {\feedetiprep}
collect threat data themselves via honeypots, analyzing malware, etc. Other
sources, such as {Badips} or the Facebook ThreatExchange, collect their
indicators from general users or organizations---\eg\ entities may be attacked
and submit the indicators to these threat intelligence services. These services
then aggregate the data and report it to their subscribers. Through this level of
aggregation the precise collection methodologies and data
providence can be lost.


\subsection{Data Source Structure}
\label{sec:feed-structure}

\ti\ sources in my corpus structure and present data in different ways.
Part of the challenge in producing cross-dataset metrics is normalizing both
the structure of the data as well as its \emph{meaning}. A major structural
difference that influences my analysis occurs between data sources that
provide data in \emph{snapshots} and data sources that provide \emph{events}.

\emphpar{Snapshot}
Snapshot feeds provide periodic snapshots of a set of indicators. More formally, a
snapshot is a set of indicators that is a function of time. It defines, for a given point in
time, the set of indicators that are members of the data source. Snapshot feeds imply
\emph{state}: at any given time, there is a set of indicators that are \emph{in} the feed.
A typical snapshot source is a published list of IPs periodically updated by its maintainer.
For example, a list of command-and-control IP addresses for a botnet may be published as a
snapshot feed subject to periodic updates.

All feeds of file hashes are snapshots and are \emph{monotonic} in the sense that indicators
are only added, not removed, from the feed. Hashes are a proxy for the file
content, which does not change (malicious file content will not change to benign in the future).

\emphpar{Event}
In contrast, event feeds report newly discovered indicators. More formally, an event source
is a set of indicators that is a function of a time \emph{interval.} For a given time interval,
the source provides a set of indicators that were seen or discovered in that time interval.
Subscribers of these feeds query data by asking for new indicators added in a recent time window.
For example, a user might, once a day, request the set of indicators that appeared
in the last 24 hours.

This structural difference is a major challenge when evaluating feeds comparatively.
I need to normalize the difference to make a fair comparison, especially for IP feeds. From a \ti\
consumer's perspective, an \emph{event} feed does not indicate when an indicator will
expire, so it is up to the consumer to act on the age of indicators. Put another way,
the expiration dates of indicators are decided by how users query the feed:
if a user asks for the indicators seen in the last 30 days
when quering data, then there is an implicit 30-day valid time window for these indicators.

In this chapter, I choose a 30-day valid period for all the indicators I collected from event feeds---the same valid period
used in several snapshot feeds, and also a common query window option offered by event feeds.
I then convert these event feeds into snapshot feeds and evaluate all of them in a unified fashion.


\begin{comment}
Snapshot sources are build on a notion of emph{state}, namely, the current indicators
in the list at a certain moment, as indicators could be added and removed from
the source over time. This type of source mainly apply for IP feeds, as IP data
is time sensitive---a malicious IP address discovered now might not be malicious
in the future. One example of such source is {\feedfeodo}\cite{Feodo}, which records
a list of current Command and Control server IPs of Feodo botnet\cite{Feodo-Tracker}
that a site should block.

Event sources, on the other hand, is ``stateless'', they focus on what have been
discovered recently. One example of this type of source is {\feednothink}\cite{Nothink}.
Upon request, it returns the source IPs they collected that have conducted SSH
brute-force attack in the last day, last week or last year, depending on the specified
query option.

This structural difference doesn't concern file hash feeds, since file hashes are time
insensitive---a malicious file hash won't change to benign in the future. One can
think of all file hash feeds as snapshot feeds where indicators never expire.
\end{comment}

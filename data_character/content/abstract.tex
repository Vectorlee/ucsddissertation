The term ``threat intelligence'' has swiftly become a staple buzzword
in the computer security industry.  The entirely reasonable premise is
that, by compiling up-to-date information about known threats (i.e.,
IP addresses, domain names, file hashes, etc.), recipients of such
information may be able to better defend their systems from future
attacks.  Thus, today a wide array of public and commercial sources
distribute threat intelligence data feeds to support this purpose.
However, our understanding of this data, its characterization and the
extent to which it can meaningfully support its intended uses, is
still quite limited.  In this paper, we address these gaps by formally
defining a set of metrics for characterizing threat intelligence data
feeds and using these measures to systematically characterize a broad
range of public and commercial sources. Further, we ground our
quantitative assessments using external measurements to qualitatively
investigate issues of coverage and accuracy. Unfortunately, our
measurement results suggest that there are significant limitations and
challenges in using existing threat intelligence data for its
purported goals.

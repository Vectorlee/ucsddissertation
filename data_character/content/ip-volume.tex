\subsection{Volume}
\label{sec:ip-volume}

%\noteby{KL}{Change this to Size and Rate.}
Volume is one of the oldest and simplest \ti\ metrics representing how informative
each data source is. Table~\ref{tab:volume-overview-1} and Table~\ref{tab:volume-overview-2} shows the total number of unique IP addresses
collected from each feed during the measurement period, under column \emph{Volume}. A \snapfeedsym\ denotes a \textit{snapshot feed}
and \deltafeedsym\ indicates an \textit{event feed} (Section~\ref{sec:feed-structure}).
\colname{Volume} is the total number of IPs collected during our measurement period.
\colname{Exclusive} is the exclusive contribution of each feed (Section~\ref{sec:ip-unique}).
\colname{Avg. Rate} is the number of average daily new IPs added in the feed (Section~\ref{sec:ip-accuracy}), and
\colname{Avg. Size} is the average working set size of each feed (Section~\ref{sec:ip-volume}).
Feeds are listed in order of decreasing volume, grouped by category.
The numbers we show are after the removal of invalid entries identified
by the sources themselves. Column \emph{Avg. Rate} shows the average number of
new IPs we received per day, and \emph{Avg. Size} lists the average daily
working set size of each feed, that is, the average size of the snapshot.

\finding\ Feeds vary dramatically in volume. Within every category, big feeds can contain
orders of magnitude more data than small feeds. For example, in the scan category, we saw
over 361,004 unique IP addresses in \feeddshield\ but only 1,572 unique addresses
in \feedTSAnalyst\ in the same time period. Clearly, volume is a major differentiator
for feeds. %In Section~\ref{sec:ip-overlap} and~\ref{sec:ip-unique}, we will examine which
%feeds duplicate each other and which feeds contribute unique indicators.

%\newcolumntype{H}{>{\setbox0=\hbox\bgroup}c<{\egroup}@{}}

\begin{table}[t!]
\centering
\caption{IP \ti\ feeds used in the study part I}
\label{tab:volume-overview-1}
\small
 \begin{tabular}{l@{}r r r r}
 \toprule
 \colname{Feed} & \colname{Volume} & \colname{Exclusive} & \colname{Avg. Rate} &  \colname{Avg. Size} \\ %& \colname{Unrt} \\
  \midrule
  \textbf{Scan Feeds} \\
  %\cline{1-1}

\snapfeedsym\  {\feedTSAlienVault}\footnote{This feed is aggregated by \emph{PA} from Alienvault OTX,
                                            the {\feedalienvault} is the public reputation feed we collected
                                            from AlienVault directly. They are different feeds.}

                                     & 425,967 	& 48.6\% 	& 1,359  & 128,821 \\
\deltafeedsym\ {\feeddshield}        & 361,004 	& 31.1\% 	& 1,556  & 69,526 \\
\snapfeedsym\  {\feedTSramnode}      & 258,719 	& 62.0\% 	& 870    & 78,974 \\
\deltafeedsym\ {\feedpacketmail}     & 246,920 	& 48.6\% 	& 942    & 29,751 \\
\snapfeedsym\  {\feedetiprep}        & 204,491 	& 75.6\% 	& 1,362  & 8,756 \\
\snapfeedsym\  {\feedTSLabScan}      & 169,078 	& 63.1\% 	& 869 	 & 9,775 \\
\snapfeedsym\  {\feedTSSnort}        & 19,085 	& 96.3\% 	& 56     & 4,000 \\
\deltafeedsym\ {\feedFBBasecamp}     & 6,066 	& 71.3\% 	& 24     & 693 \\
\snapfeedsym\  {\feedTSAnalyst}      & 1,572 	& 34.5\% 	& 6.3 	 & 462 \\


  %\midrule
  \textbf{Botnet Feeds} \\
  %\cline{1-1}
\snapfeedsym\  {\feedTSAnalyst}     & 180,034 	& 99.0\% 	& 697 	    & 54,800 \\
\snapfeedsym\  {\feedTSCI}          & 103,281 	& 97.1\% 	& 332 	    & 30,388 \\
\snapfeedsym\  {\feedetiprep}       & 77,600 	& 99.9\% 	& 567 	    & 4,278 \\
\snapfeedsym\  {\feedTSBotscout}    & 23,805 	& 93.8\% 	& 81 	    & 7,180 \\
\snapfeedsym\  {\feedTSVoIP}        & 10,712 	& 88.0\% 	& 40 	    & 3,633 \\
\snapfeedsym\  {\feedTSCompr}       & 7,679 	& 87.0\% 	& 21 	    & 2,392 \\
\snapfeedsym\  {\feedTSBots}        & 4,179 	& 80.7\% 	& 16 	    & 1,160 \\
\snapfeedsym\  {\feedTSHoneypot}    & 2,600 	& 86.5\% 	& 8.5 	    & 812 \\


  %\midrule
  \textbf{Brute-force Feeds} \\
  %\cline{1-1}
\deltafeedsym\  {\feedbadipssh}     & 542,167 	& 84.1\% 	& 2,379 	& 86,677 \\
\deltafeedsym\  {\feedbadipbot}     & 91,553 	& 70.8\% 	& 559 	    & 17,577 \\
\snapfeedsym\   {\feedetiprep}      & 89,671 	& 52.8\% 	& 483 	    & 3,705 \\
\snapfeedsym\   {\feedTSBrute}      & 41,394 	& 92.1\% 	& 138 	    & 14,540 \\
\deltafeedsym\  {\feedusername}     & 37,198 	& 54.2\% 	& 179 	    & 3662.8 \\
\deltafeedsym\  {\feeddisco}        & 31,115 	& 43.6\% 	& 40 	    & 1,224 \\
\deltafeedsym\  {\feedFBZendesk}    & 22,398 	& 77.3\% 	& 74 	    & 2,086 \\
\deltafeedsym\  {\feednothink}      & 20,325 	& 62.7\% 	& 224 	    & 12,577 \\
\deltafeedsym\  {\feeddangerrule}   & 10,142 	& 4.88\% 	& 37	    & 1,102 \\

  %\midrule
  \textbf{Malware Feeds} \\
  %\cline{1-1}
\snapfeedsym\  {\feedetiprep} 	     & 234,470 	& 99.1\% 	& 1,113 	& 22,569 \\
\deltafeedsym\ {\feedFBAdmin} 	     & 30,728 	& 99.9\% 	& 129 	    & 3,873 \\
\snapfeedsym\  {\feedfeodo} 	     & 1,440 	& 47.7\% 	& 1.3 	    & 1,159 \\
\snapfeedsym\  {\feedTSLabMalware}   & 1,184 	& 84.6\% 	& 3.5 	    & 366 \\
\deltafeedsym\ {\feedmalcode} 	     & 865 	    & 61.0\% 	& 2.9 	    & 86.6 \\
\snapfeedsym\  {\feedTSBambenek}     & 785 	    & 92.1\% 	& 3.4 	    & 97.9 \\
\snapfeedsym\  {\feedTSSSL} 	     & 676 	    & 53.9\% 	& 2.9 	    & 84.0 \\
\snapfeedsym\  {\feedTSAnalyst}	     & 492 	    & 79.8\% 	& 2.1 	    & 149 \\
\snapfeedsym\  {\feedTSAbusech}      & 256 	    & 7.03\% 	& 1.6 	    & 117 \\
\snapfeedsym\  {\feedTSMalTraffic}   & 251 	    & 60.5\% 	& 0.9 	    & 72 \\
\snapfeedsym\  {\feedzeus}           & 185 	    & 49.1\% 	& 0.5 	    & 101 \\
\bottomrule
\end{tabular}
\end{table}


\begin{table}[t!]
\centering
\caption{IP \ti\ feeds used in the study part II}
\label{tab:volume-overview-2}
\small
 \begin{tabular}{l@{}r r r r}
 \toprule
 \colname{Feed} & \colname{Volume} & \colname{Exclusive} & \colname{Avg. Rate} &  \colname{Avg. Size} \\ %& \colname{Unrt} \\
  \midrule
  %\midrule
  \textbf{Exploit Feeds} \\

\deltafeedsym\  {\feedbadiphttp}    & 305,020 	& 97.6\% 	& 1,592 	& 22,644 \\
\deltafeedsym\  {\feedbadipftp}     & 285,329 	& 97.5\% 	& 1,313 	& 27,601 \\
\deltafeedsym\  {\feedbadipdns}     & 46,813 	& 99.3\% 	& 231 	& 4,758 \\
\deltafeedsym\  {\feedbadiprfi}     & 3,642 	& 91.4\% 	& 16 	& 104 \\
\deltafeedsym\  {\feedbadipsql}     & 737 	    & 79.5\% 	& 4.4 	& 99.2 \\

 \textbf{Spam Feeds} \\
  %\cline{1-1}
\snapfeedsym\  {\feedetiprep}       & 543,583 	& 99.9\% 	& 3,280 	& 6,551 \\
\deltafeedsym\ {\feedbadippostfix}  & 328,258 	& 90.5\% 	& 842 	    & 27,951 \\
\deltafeedsym\ {\feedbadipspam}     & 302,105 	& 89.3\% 	& 1,454 	& 30,197 \\
\snapfeedsym\  {\feedTSBotscout}    & 14,514 	& 89.3\% 	& 49 	& 4,390 \\
\snapfeedsym\  {\feedalienvault}    & 11,292 	& 96.6\% 	& 48 	& 1,328 \\

\bottomrule
\end{tabular}
\end{table}

%\footnotetext{}



Average daily rate represents the amount of new indicators collected from a feed each day.
Some feeds may have large volume but low daily rates, like {\feedfeodo} in the malware
category. This means most indicators we get from that feed are old data present in the feed before our
measurement started. On the other hand, the average rate of a feed could be
greater than the volume would suggest, like {\feednothink} in the brute-force category.
This is due to the fact that indicators can be added and removed multiple times in a
feed. In general, IP indicators tend to be added in a feed only once: 37 among \numipfeeds\
IP feeds have over 80\% of their indicators appearing only once, and 30 of them have this rate over 90\%.
One reason is that some snapshot feeds maintain a valid period for each indicator, as
we found in all \emph{PA} feeds where the expiration date of each indicator is explicitly
recorded. When the same indicator is discovered again by a feed before its expiration time,
the feed will just extend its expiration date, so this occurrence will not be captured
if we simply subtract the old data from the newly collected data to derive what is added on a day.
For event feeds and snapshot feeds in \emph{PA} where we can precisely track every occurrence
of each indicator, we further examed data occurrence frequency and still found that the vast
majority of IPs in feeds only occurred once---an observation that relates to the dynamics
of cyber threats themselves.

\feednothink, as we mentioned above, is a notable exception. It has over 64\% of its
indicators appearing 7 times in our data set. After investigating, we
found that this feed posts all its previous data at the end of every month, behavior very likely
due to the feed provider instead of the underlying threats.
% {\feedusername}, {\feedbadipftp}, {\feedbadipdns} and {\feedbadipsql}
% all have a large percent of IPs appeared twice, and we found
% that all 4 feeds reported a abnormally large amount of same IPs on 2018 July 1st and July 13th.
% We will discuss more about this in Section~\ref{sec:ip-accuracy}.

The working set size defines the daily average amount of indicators users need to store in their
system to use a feed (the storage cost of using a feed). The average
working set size is largely decided by the valid period length
of the indicators, controlled either
by the feed (snapshot feeds) or the user (event feeds). The longer the valid period is,
the larger the working set will be. Different snapshoot feeds have different choices for this
valid period: {\feedTSAlienVault} in the scan category sets a 90-day valid period for
every indicator added to the feed, while {\feedTSAbusech} uses a 30-day period. Although we do not
know the data expiration mechanism used by snapshot feeds other than \emph{PA} feeds, as
there is no related information recorded, we can still roughly estimate this by checking the
\emph{durations} of their indicators---the time between an indicator being added and being removed.
Four {\feedetiprep} feeds have more than 85\% of durations shorter than 10 days, while the one in
the malware category has more than 40\% that span longer than 20 days. {\feedfeodo} has over 99\% of
its indicators valid for our entire measurement period, while over 70\% of durations in the
{\feedzeus} are less than 6 days. We did not observe a clear pattern regarding how each snapshot
feed handles the expiration of indicators.

%The column \emph{Occurred} presents the number of unique IPs added in a \ti\ source after our
%measurement period started. We differentiate \emph{Volume} and \emph{Occurred} since IP
%indicators are time sensitive: They are added, removed and could be added again in a feed over
%time. When we pull data from a snapshot feed on a day, some IPs are actually added
%on previous days and are still kept by the feed. \emph{Occurred} only captures the IPs that
%are newly added after the measurement period started, while \emph{Volume} includes legacy data in
%a feed when we collect indicators on the first day of our analysis. At a high level,
%\emph{Volume} represents the number of IPs that a feed \emph{considers} as malicious during
%a period of time, and \emph{Occurred} represents the number of newly discovered indicators by the
%feed during that time.

%For delta feeds, we calculated their \emph{Volume} by tracking the expiration dates of the
%indicators we collected before (The expiration dates are provided in the metadata) and included
%the indicators whose expiration dates are after January 1st, 2016.

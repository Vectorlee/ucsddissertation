\subsection{Phishing Detection}
Phishing attacks is a type of social engineering attack, where attackers 
disguise as trusted entities and lure victims to click on malicious links, 
download attachments, or provide sensitive information like credit card
numbers or online credentials. These attack can have devastating results. 
For individuals, this includes unauthorized access to their online 
accounts, the stealing of money in their bank, or identify theft for illegal
activities. Moreover, phishing is often used to gain a foothold in corporate 
or governmental networks as a part of a larger attack, such as an 
advanced persistent threat (APT) event. This could result in significant
financial losses, leakage of user data and protected technologies. Hence,
phishing related data is an important part of threat intelligence. These data
covers phishing domains and URLs, as well as IP addresses that host these
websites. 

Phishing attacks are often launched through a phish email, so researchers
have looked into different methods to identify these phish emails. Phish emails
are quite different spam emails. Spam emails are usually easy to identify,
since they just try to advertise certain goods and don't try to conceal 
their identities. Phish emails, however, can be hard to distinguish by 
everyday people, since these emails deliberately pretend they are
from legitimate sources, and they use multiple tactics to lure recipients
to click on the link. For example, attackers exploit HTML emails and
provide a HREF where the actually link is different from what is being 
displayed, like \texttt{<a href="badsite.com">paypal.com</a>}. Therefore,
when analyzing if an email is a phish email, researchers uses text content
of the email (like in spam email detection) together with phishing specific 
features, like the one listed above, to build machine learning model. 
Examples works such as Fette et. al~\cite{fette2007learning}, Abu-Nimeh 
et. al~\cite{abu2007comparison}, and Chandrasekaran et. al
~\cite{chandrasekaran2006phishing}.

Since phishing attacks mostly require victims to click on a link, the URLs 
and the corresponding webpages also provide us clues from which we can 
determine whether it is phish related. For example, in early days, many
phish URL will have IP address in it, like \texttt{http://135.12.44.20/index.php}. This feature is extremely rare for
legitimate websites. Another characteristic is that phish domains tend to
relatively long, containing multiple segments. For example, 9794.myonlineaccounts2.abbeynational.co.uk.syrialand.com. The idea is that
a very long domain can confuse people and sometimes people only read the 
beginning of a domain and think it is legitimate~\cite{wu2006security}.
Whittaker et. al. from Google has experimented with multiple URL related
features and implemented them in their machine learning algorithm
~\cite{whittaker2010large}. They also used other URL metadata like PageRank
to assist the detection. Another source to look at is the phishing
webpage. Phishing attacks usually try to mimic other trusted websites and
make people think they are visiting the original legitimate websites. They
also frequently ask people to type their credentials or provide payment
information, since the goal of the attackers is to steal these information.
Researchers have used the content of the webpages as the feature to further
enhance their algorithms. Zhang et. al.~\cite{zhang2007cantina} has used 
the word frequency on the webpage(TF-IDF) to classify the webpages. Wardman 
et. al.~\cite{wardman2008automating} compared the content on target webpage
with other popular webpages to see if the webpages in question are trying
to mimic those legitimate ones. All these resources provide additional
confidence to people when deciding if a URL is a phish URL.
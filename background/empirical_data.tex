\section{Threat Intelligence Data Study}
\label{sec:threat_intel_data}

Previous sections cover the techniques people had proposed to generate
threat intelligence data and sharing them. With these 
techniques in place, what does the real-world threat intelligence 
data look like, is another interesting direction to study. Although 
individual threat detection techniques tend to promise high accuracy and
coverage on target threats, after being deployed in real-world for a long
time, the final data product might be far different from what 
people originally expected. Empirically study the characteristic and 
performance of threat intelligence data is an important approach to 
understand the current status, and thus provides insights for future 
innovations.

Several studies have examined the effectiveness of blacklist-based 
threat intelligence~\cite{kuhrer2014paint, ramachandran2006revealing, 
ramachandran2007filtering, sheng2009empirical, sinha2008shades}.
Ramachandran~\etal~\cite{ramachandran2007filtering} showed that spam 
blacklists are both incomplete (missing 35\% of the source IPs of 
spam emails captured in two spam traps), and slow in responding 
(20\% of the spammers remain unlisted after 30 days).
Sinha~\etal~\cite{sinha2008shades} further confirmed this result by 
showing that four major spam blacklists have very high false negative
rates, and analyzed the possible causes of the low coverage.
Sheng~\etal~\cite{sheng2009empirical} studied the effectiveness of
phishing blacklists, showing the lists are slow in reacting to
highly transient phishing campaigns.

Other studies have analyzed the general attributes of threat
intelligence data. Pitsillidis~\etal~\cite{tasters:imc12} studied the
characteristics of spam domain feeds, showing different perspectives
of spam feeds, and demonstrated that different feeds are suitable for
answering different questions. Thomas~\etal~\cite{thomas2016abuse}
constructed their own threat intelligence by aggregating the abuse
traffic received from six Google services, showing a lack of
intersection and correlation among these different sources. 

The limitations of the previous measurement works are that these 
studies tend to only focused on specific types of threat intelligence 
sources, like spam or phish blacklists, and they only evaluated one 
aspect of the data characteristic, like the operational performance, 
rather than generalize the measurement and define threat intelligence 
metrics that can be extended beyond the work.

Little work before had defines a general measurement methodology to
examine threat intelligence across a broad set of types and categories.
Metcalf~\etal~\cite{metcalf2015blacklist} collected and measured IP
and domain blacklists from multiple sources, but again only focused 
on volume and intersection analysis. One missing piece in these works
is that they did not approach the problem from the perspective of 
consumers of threat intelligence.  After all, it is the consumers that
will support this industry, and research communities should look more
into their needs. This is one major motivation of my work, which will
be discussed in Chapter~\ref{chapter:data_character}.
\section{Empirical Analysis on Threat Intelligence Usage}

However, despite all the excitement in industry and acamedia, there is one
important aspect of threat intelligence that little work has looked into before:
how threat intelligence products are being used by organizations at a large scale.
Understanding this question gives us ideas about the adoption of threat
intelligence data, which is an essential information to know in the threat
intelligence ecosystem. More importantly, the usage of different produts offers
us insight into the potential impact they could cost on the Internet.
As recent work has shown~\cite{li2019reading}, false positives are relatively
common in threat intelligence feeds. If an organization is using one threat
intelligence IP feed in its firewall for IP-based traffic blocking, and there
is a false positive(a benign IP address) in this feed, then all the users in
that organization will be affected. From another side, if one online host is
added to an IP feed mistakenly, then this host will lose access to all the
organizations that use this feed as a blocking ruleset. Therefore, the usage
problem is a crucial topic that should get more attention from security community.

But this problem is also a very challenging problem. The primary challenge 
is that there are many ways people can use these data. It can be directly 
used to block network traffic, or just raise an alarm in their Security
Information and Event Management(SIEM) systems. Some use cases do not even 
have a well-defined behavior that we can quantitatively measure. 
The actions derived from threat intelligence data can also happen at different
network layer. For example, an organization can deny access on network layer; 
it can also deny access on application layer, like HTTP 403 Forbidden. 
This diverse possibility of use cases make it hard to assess this problem as 
a whole. Another challenge for us academic researchers is that we do not have 
access to the commerical threat intelligence data, as they can be prohibitively
expensive. Therefore, it is hard to capture the comprehensive picture on the
Internet.

Little work has been done to try and understand how threat
intelligence data is being used on a large scale. The only work that tries to
look at threat intelligence data from this perspective are industrial surveys
-- wherein organizations fill out questionnaires. One such survey conducted by
the Ponemon Institute~\cite{ponemon2018cti}, surveyed 1,200 IT and IT
security practitioners, asking if they use threat intelligence products, and
if they do what tools do they use that utilize the threat intelligence data.
The survey also asked their user experience. SANS Institute did a
similar study~\cite{shackleford2017cyber} where they surveyed 600
participants from a diverse industry background and asked questions about
their threat intelligence usage. These works are limited in terms
of scale, and their results are all in a very high-level. Although they offered
some useful insight, they can not provide us a concrete understanding
about the use of threat intelligence data on a large scale. My measurement,
which will be discussed in Chapter~\ref{chapter:data_usage},
is the first work that systematically looks at the problem of inferring use
of threat intelligence data on a large scale, and explore its implications.

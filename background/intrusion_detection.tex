\subsection{Intrusion Detection}

One direct way to capture threat is to identify the attacker when a
system in use is under attack or compromised, so we can capture the 
attacker in action. The threat actors recorded this way will not be 
many, but the data would have very high accuracy, since they are
directly recorded from real attacks. The technique we use here is also 
the technique to protect the system in the first place:
\textit{Intrusion Detection}.

Intrusion Detection aims to detect attacks on network or system on the
fly. The techniques involved can generally be classified as two 
categories: misuse-based detection and anomaly-based detection.
Misuse-based approaches utilize pre-defined patterns and signatures
of malicious behaviors, and dynamically compare the behavior of the
system against these patterns to spot potential intrusion. On the another
hand, anomaly-based detection construct a model for the normal behavior 
of a system, then check if the current behavior is deviating from the 
``normal'' behavior.

The performance of misuse-based detection methods depends on the quality
of the pre-defined attack patterns (or detect policies), and these 
patterns are usually provided manually by security experts. Since 
different attack vectors vary a lot on their approaches and system 
component they touch, the 
patterns provided by the detection system must be able to cover a 
diverse set of behavior, also to be extendable, as new attacking
methods are keep showing up. Therefore, early work on this technique like
Bro~\cite{paxson1999bro}, Snort~\cite{roesch1999snort}, 
P-BEST~\cite{lindqvist1999detecting}, and STAT~\cite{vigna2003designing}
all emphasize on the providing a powerful threat modeling language, which
can express a broad set of threat, also easy to program to include new 
patterns. Recent works like~\cite{bugiel2012towards} move to new
system environment (e.g. mobile system like Android), but still focus
on providing a expressive modeling language to build detect policies. 
These misuse-based detection methods is generally accurate, since the
pre-defined attack patterns are usually well defined by security experts. 
The main disadvantage of this technique is that it can only detect 
modeled attacks. For new type of attacks where there are no pattern
written, misuse-based system will miss them completely.

As a complementary, anomaly-based detection methods try to detect if
the behavior of an application or system is different from its benign
behaviors. This technique relies on modeling the normal behavior of a 
system in question, and since there are so many possible things a program 
can do, we need to simplify the ``behavior'' to precisely reason about 
it. Forrest et. al.~\cite{forrest1996sense} first proposed to use
syscall sequence of a program as its behavior, since a program can only
affect the operating system with syscalls, and from the security 
perspective, these would be the only behaviors that matters. Follow up
works like ~\cite{lee1998data, warrender1999detecting, mutz2006anomalous}
explored different data model to better distinguish abnormal sequences
from benign ones, using method like data mining, Bayesian network and 
neural network. Recent works like~\cite{du2017deeplog} also try to 
utilize more data sources and experiment with more advanced machine
learning algorithm for the detection. The advantage of these approaches 
is that it can capture previously unknown attacks. However, it tends
to suffer more false positives, since a normal program can show abnormal
behaviors once a long while, and it is very hard to capture these
cases when we first build the ``normal'' behavior set for a program.

\subsection{Causality Analysis}
Intrusion detection techniques described before detects threats 
when abnormal behavior is observed. But for many advanced 
attacks, especially the APTs(Advanced Persistent Threats), it 
is not always obvious to trace from the malicious behavior, e.g.
a running malware, to the source of the attacks, e.g. a phish 
URL that distributes the malware. This is mainly because sophisticated
attacks often take precautions to hide their traces by deleting
system or application logs, they can also prolong the attacks, 
creating a large window from the time they break into victims' 
machines till the time they actually carry out malicious activities. 
All of these can create difficulties for administrators to diagnose
the intrusion and trace the source. In the context of Threat 
Intelligence, it is important to uncover the source of 
an attack, so we can provide valuable intelligence for the community 
to defense the same attack in an early stage.

The analysis, which tracks causal relationships between files and 
processes to diagnose attack provenances and consequences, is called
\textit{Attack Causality Analysis}. It is an critical technique
during Threat Intelligence hunting. The pioneering work in this field
is done by King and Chen~\cite{king2003backtracking}. They first defined
the event dependency graph, where the nodes represent processes and files
and edges represent the events between process and files, for example,
a process spawn another process, or a process read or write to a 
file. Given a detection point, like a suspicious files or a running
process, the system builds a dependency graph from this points by
processing event logs, then using the timestamp of each events and 
their causal relationship, we can trace back to the source and identify
the original intrusion point.

A simple idea as it seems, it captured the fundamental logic behind 
causality analysis: information flow tracking. However, this simple
solution quickly runs into a problem in a complex system: 
\textit{dependence explosion}~\cite{goel2005taser}, where there are 
so many processes and files involved in this dependency graph, together
with large amount inter-relations, that it is very hard identify the
real attack from the haystack. The case become even more true when there
are long-running programs, like a server program. The dependency 
associated with these program will grow enormously over 
time~\cite{lee2013high}, making the dependency graph even more 
complicated.

Many work have tried to tackle this problem. Some heuristics have been 
proposed to prune the graph, like in~\cite{king2005enriching}, the 
authors utilized the fact that worms try to exploit from host to host,
so the traffic from a host who already has an IDS alert is more likely
associated with worm attacks. Liu et al.~\cite{liu2018towards} uses
the rareness of events as a metric to prioritize searching on dependency
graphs. Other work try to break the entities on the graph into smaller 
granularity, so we can pinpoint the causal relationship between objects.
For example, Goel et al.~\cite{goel2005taser} use the separate socket
reads to partition the execution of a program into different segments, 
so the monitoring system can figure out which action of that program
is corresponded with which exact network request. Lee et 
al.~\cite{lee2013high} use the prevalence of event-loops in programs to
partition the execution of the program based on each loop iteration,
and then associates events with specific loop iterations. Binary taint
tracking has also been utilize to provide richer semantic information, 
like demonstrated in~\cite{ma2016protracer}. This topic is still an 
active topic today and researcher are trying to solve this from different
angle.
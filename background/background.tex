\chapter{Background}
\label{chapter:background}

\section{Threat Hunting}
\textit{Threat Hunting} is the process of generating threat intelligence
data. Cyber-attacks are happening very day on the Internet, with vantage 
points, data providers can effectively capture these attacks and generate 
indicator-of-compromises. These vantage points can be specially
deployed infrastructures, including honeypot and Internet telescope; data
providers can also directly collect data from from end-users, like 
anti-virus vendors collecting potential malicious executable from their
customers. 

However, as one can
imagine, the raw data usually contains a lot of noises and can hardly
serve as meaningful ``intelligence''. Threat hunting aims to use domain 
knowledge and specialized algorithms to detect threats and identify
attackers from the raw data, therefore providing real valuable 
information about different attacks. By definition, this process involves
detecting all possible attacks and covers all detection oriented 
techniques. In this chapter, I will go through previous research works
on several representative threat categories and discuss the techniques to 
detect these threats. These works fall into the algorithmic direction on
threat intelligence data, base on the definition in 
Chapter~\ref{chapter:introduction}.

\subsection{Intrusion Detection}

One direct way to capture threat is to identify the attacker when a
system in use is under attack, so we can capture the 
attacker in action. These systems can be specially deployed systems
just for luring attackers, like honeypot; they can also be real systems
with real users, where people deploy detection system and capture 
attacks when they are happening. The technique we use here is also 
the technique to protect the system in the first place:
\textit{Intrusion Detection}.

Intrusion detection aims to detect attacks on network or system on the
fly. The techniques involved can generally be classified as two 
categories: misuse-based detection and anomaly-based detection.
Misuse-based approaches utilize pre-defined patterns and signatures
of malicious behaviors, and dynamically compare the behavior of the
system against these patterns to spot potential intrusion. On the another
hand, anomaly-based detection construct a model for the normal behavior 
of a system, then check if the current behavior is deviating from the 
``normal'' behavior.

The performance of misuse-based detection methods depends on the quality
of the pre-defined attack patterns (or detect policies), and these 
patterns are usually provided manually by security experts. Since 
different attack vectors vary a lot on their approaches and system 
component they touch, the 
patterns provided by the detection system must be able to cover a 
diverse set of behavior, also to be extendable, as new attacking
methods are keep showing up. Therefore, early work on this technique like
Bro~\cite{paxson1999bro}, Snort~\cite{roesch1999snort}, 
P-BEST~\cite{lindqvist1999detecting}, and STAT~\cite{vigna2003designing}
all emphasize on the providing a powerful threat modeling language, which
can express a broad set of threat, also easy to program to include new 
patterns. Recent works like~\cite{bugiel2012towards} move to new
system environment (e.g. mobile system like Android), but still focus
on providing a expressive modeling language to build detect policies. 
These misuse-based detection methods is generally accurate, since the
pre-defined attack patterns are usually well defined by security experts. 
The main disadvantage of this technique is that it can only detect 
modeled attacks. For new type of attacks where there are no pattern
written, misuse-based system will miss them completely.

As a complementary, anomaly-based detection methods try to detect if
the behavior of an application or system is different from its benign
behaviors. This technique relies on modeling the normal behavior of a 
system in question, and since there are so many possible things a program 
can do, we need to simplify the ``behavior'' to precisely reason about 
it. Forrest et. al.~\cite{forrest1996sense} first proposed to use
syscall sequence of a program as its behavior, since a program can only
affect the operating system with syscalls, and from the security 
perspective, these would be the only behaviors that matters. Follow up
works like ~\cite{lee1998data, warrender1999detecting, mutz2006anomalous}
explored different data model to better distinguish abnormal sequences
from benign ones, using method like data mining, Bayesian network and 
neural network. Recent works like~\cite{du2017deeplog} also try to 
utilize more data sources and experiment with more advanced machine
learning algorithm for the detection. The advantage of these approaches 
is that it can capture previously unknown attacks. However, it tends
to suffer more false positives, since a normal program can show abnormal
behaviors once a long while, and it is very hard to capture these
cases when we first build the ``normal'' behavior set for a program.

\subsection{Causality Analysis}
Intrusion detection techniques described before detects threats 
when abnormal behavior is observed. But for many advanced 
attacks, especially the APTs(Advanced Persistent Threats), it 
is not always obvious to trace from the malicious behavior, e.g.
a running malware, to the source of the attacks, e.g. a phish 
URL that distributes the malware. This is mainly because sophisticated
attacks often take precautions to hide their traces by deleting
system or application logs, they can also prolong the attacks, 
creating a large window from the time they break into victims' 
machines till the time they actually carry out malicious activities. 
All of these can create difficulties for administrators to diagnose
the intrusion and trace the source. In the context of Threat 
Intelligence, it is important to uncover the source of 
an attack, so we can provide valuable indicator-of-compromise for the 
community to defense the same attack in an early stage.

The analysis, which tracks causal relationships between files and 
processes to diagnose attack provenances and consequences, is called
\textit{Attack Causality Analysis}. It is an critical technique
during threat intelligence hunting. The pioneering work in this field
is done by King and Chen~\cite{king2003backtracking}. They first defined
the event dependency graph, where the nodes represent processes and files
and edges represent the events between process and files, for example,
a process spawn another process, or a process read or write to a 
file. Given a detection point, like a suspicious files or a running
process, the system builds a dependency graph from this points by
processing event logs, then using the timestamp of each events and 
their causal relationship, we can trace back to the source and identify
the original intrusion point.

A simple idea as it seems, it captured the fundamental logic behind 
causality analysis: information flow tracking. However, this simple
solution quickly runs into a problem in a complex system: 
\textit{dependence explosion}~\cite{goel2005taser}, where there are 
so many processes and files involved in this dependency graph, together
with large amount inter-relations, that it is very hard identify the
real attack from the haystack. The case become even more true when there
are long-running programs, like a server program. The dependency 
associated with these program will grow enormously over 
time~\cite{lee2013high}, making the dependency graph even more 
complicated.

Many work have tried to tackle this problem. Some heuristics have been 
proposed to prune the graph, like in~\cite{king2005enriching}, the 
authors utilized the fact that worms try to exploit from host to host,
so the traffic from a host who already has an IDS alert is more likely
associated with worm attacks. Liu et al.~\cite{liu2018towards} uses
the rareness of events as a metric to prioritize searching on dependency
graphs. Other work try to break the entities on the graph into smaller 
granularity, so we can pinpoint the causal relationship between objects.
For example, Goel et al.~\cite{goel2005taser} use the separate socket
reads to partition the execution of a program into different segments, 
so the monitoring system can figure out which action of that program
is corresponded with which exact network request. Lee et 
al.~\cite{lee2013high} use the prevalence of event-loops in programs to
partition the execution of the program based on each loop iteration,
and then associates events with specific loop iterations. Binary taint
tracking has also been utilize to provide richer semantic information, 
like demonstrated in~\cite{ma2016protracer}. This topic is still an 
active topic today and researcher are trying to solve this from different
angle.
\subsection{Malware Analysis}
Malicious software, often called malware, is always a pressing threat 
on the Internet. It has been used by attackers to steal sensitive user data, 
control victim machines to launch spam campaigns or DDoS campaigns, or 
encrypt valuable data to demand ransom, etc. Symantec has reported over 200 
Million new malware variants just in 2018 alone~\cite{symantecmalware}. 
Therefore, it is crucial for threat intelligence products to cover recent 
malware comprehensively.
Malware threat intelligence data usually comes as two forms: file hashes,
like MD5, that represent malware variants themselves, and IPs or domains
that host Command-and-Control servers for the malware. Both forms of 
data is critical for organizations, as the presence of either form 
indicates a strong possibility of compromise, and immediate actions need 
to be taken. To generate these data, security companies rely on analyzing 
unknown binaries, collected from the Internet or uploaded by customers,
to determine if they are malicious.

To identify if an unknown binary is malware, one straightforward yet
effective way is to check if the binary is a variant of known malware.
Since it is nontrivial to develop a sophisticated malware program, 
attackers tend to just modify existing malware to generate new unseen
variants. Some typical ways include code transformation(e.g. replace ``mov 
eax, 0'' with ``xor eax, eax''), obfuscation, or encrypting the original 
binary and stores the result as data in a new executable (with a packer 
program). These simple techniques enable attackers to quickly generate a 
large number of variants from a single malware instance, and significantly 
increase the overhead for security experts to analyze them.

To compare a new malware sample with existing ones, we need to define a 
suitable representation of malware samples, from which we can calculate 
the similarity between them. There are two approaches to construct this 
``representation'': \textit{Content-based} and \textit{Behavior-based}. 
Content-based approach abstracts a program based on its code content, and 
calculate the similarity between programs by comparing their code. 
Early work by M. Gheorghescu~\cite{Gheorghescu2006ANAV} propose 
to break the malware program into basic blocks and compare the similarity 
between those blocks.
Dullien et al.~\cite{dullien2005graph} extract control flow graphs from 
malware programs and use graph similarity as the similarity between programs.
These content-based approaches rely on analyzing program code itself, so 
they still suffer the problem of advanced code obfuscation, which can modify
the code dramatically while maintaining the same functionality. This leads to
the behavior-
based approach, where we extract the actual behavior of malware and use that
as the signature for comparison. Lee et al.~\cite{lee2006behavioral} propose
to use system call sequence as the signature to classify different malware
samples. Bailey et al.~\cite{bailey2007automated} use the \textit{non-transient
state changes} malware causes on a system(files written, processes created) as 
the behavior signature, and do the comparison based on these behaviors. Holz 
et al.~\cite{rieck2008learning} further developed this behavior-based method, 
and use the actions of malware as machine learning features, and use a 
supervised machine learning model to conduct the comparison and classification.
Bayer at al.~\cite{bayer2009scalable} used taint tracking to capture a finer
granularity of malware's behavior, and use this information for more precise
identification. The behavior-based approaches usually capture the behavior of
malware through dynamically executing the malware samples, so it won't be 
affected by malware code itself, but executing the code for every variant in
question impose nontrivial overhead. 

When there is no existing malware to compare with, or the program in question 
does not match any known malware, an analysis system will need to decide if
the program is malicious just based on the behavior of the program itself. Like
the intrusion detection methods described in the previous section, the logic
here also relies on having ``specifications'' that cover potential malware 
behaviors, and check if the analyzed program exhibits those behaviors. One 
common heuristic is to check if the program makes any changes to the system
registry. GateKeeper~\cite{wang2004gatekeeper}, for example, detect spyware
by monitoring if the program register as an OS auto-start extension, such as 
an NT service, a tray icon in Windows, or a Unix daemon/cron job. Other tools
also check different detection points, like VICE~\cite{bulter2004vice}, which
checks for the existence of various hooks used by rootkits. More advanced 
systems tend to further monitor the detailed behavior of the program, like 
in~\cite{kirda2006behavior}, the authors try to detect a popular type of
spyware that uses Internet Explorer’s Browser Helper Object (BHO) and 
toolbar interfaces to monitor a user’s browsing behavior. The system uses 
dynamic analysis to track if the program monitors users' actions and sends
out its findings to an external entity. Panorama~\cite{yin2007panorama}, 
similarly, use dynamic taint tracking to construct the information flow of
an unknown program, and then use pre-defined policies(specifications) to 
determine if the program is malicious or not.
\subsection{Spam Detection}
Spam email, also referred as unsolicited bulk email or junk mail, is 
Internet mail that is sent to a group of recipients who have no intention 
to receive it. They are particularly harmful, as these emails jams users' 
mailboxes, engulf important personal mail, waste network bandwidth and 
can even crash mail-servers. Spam emails also serve as an important
way of advertising products in underground market, like prescription drugs, 
illegal porn, replica of other brands etc. It is an old yet still popular
threat on the Internet. Threat intelligence spam data contains email address 
used by the spammers, domains and IP addresses where spammers' mail servers 
are located. Organizations, and even ISPs, tend to use these data to block 
the incoming mail traffic.

Email providers, ISPs or anti-spam organizations usually set up 
\textit{Spamtrap} to capture spam emails. Spamtrap are the email addresses
that are created not for communication, but rather to lure spam. These
email addresses do not belong to any person and will not involve in any
kind of communication. These addresses will not be revealed openly online,
so only unsolicited spammers, who tend to collect target email addresses
by crawling the Internet, or go through all possible lexical combinations
for email names, will hit these addresses. The Spamtrap here serve as a 
bait to capture spammers. People also recycle long out-of-date email 
addresses as Spamtrap addresses. 

However, simply regarding all emails received in Spamtrap as spam emails 
will create a substantial amount of false positives, since legitimate 
senders with poor data hygiene or acquisition practices end up hit the traps 
as well. To further distinguish spam emails and consequently identify the
senders, one needs to look at the content of emails themselves. Numerous 
statistical algorithms have been proposed by researchers to filter spam
emails from legitimate ones. At its core, spam filtering can be viewed as
a text categorization task: given the full text content of an email, 
decide whether it is spam email or benign email. A variety of supervised
machine learning techniques have been tested for spam filtering. Like the
naive Bayes classifier~\cite{androutsopoulos2000evaluation, sahami1998bayesian, schneider2003comparison}, 
RIPPER rule induction algorithm~\cite{cohen1996learning},
Support Vector Machine~\cite{drucker1999support}, memory-based learning
~\cite{androutsopoulos2000learning}, AdaBoost~\cite{carreras2001boosting},
and maximum entropy model~\cite{zhang2003filtering}. These algorithms all
convert email headers and body into features for the machine learning model, 
the ``bag-of-words'' approach,
and different feature reduction and weight assignment strategy have been
explored. These models can all achieve decent accuracy, and have been tested
and deployed in real-world. 

Email providers also get help from the customers to identify spams. Since
customers can mark an email as spam manually, having a large customer base
enables the email provider to collect a large amount of spam emails with high
accuracy, and therefore track down the domains and IP addresses used by the 
spammers.
\subsection{Phishing Detection}
Phishing attacks is a type of social engineering attack, where attackers 
disguise as trusted entities and lure victims to click on malicious links, 
download attachments, or provide sensitive information like credit card
numbers or online credentials. These attack can have devastating results. 
For individuals, this includes unauthorized access to their online 
accounts, the stealing of money in their bank, or identify theft for illegal
activities. Moreover, phishing is often used to gain a foothold in corporate 
or governmental networks as a part of a larger attack, such as an 
advanced persistent threat (APT) event. This could result in significant
financial losses, leakage of user data and protected technologies. Hence,
phishing related data is an important part of threat intelligence. These data
covers phishing domains and URLs, as well as IP addresses that host these
websites. 

Phishing attacks are often launched through a phish email, so researchers
have looked into different methods to identify these phish emails. Phish emails
are quite different spam emails. Spam emails are usually easy to identify,
since they just try to advertise certain goods and don't try to conceal 
their identities. Phish emails, however, can be hard to distinguish by 
everyday people, since these emails deliberately pretend they are
from legitimate sources, and they use multiple tactics to lure recipients
to click on the link. For example, attackers exploit HTML emails and
provide a HREF where the actually link is different from what is being 
displayed, like \texttt{<a href="badsite.com">paypal.com</a>}. Therefore,
when analyzing if an email is a phish email, researchers uses text content
of the email (like in spam email detection) together with phishing specific 
features, like the one listed above, to build machine learning model. 
Examples works such as Fette et. al~\cite{fette2007learning}, Abu-Nimeh 
et. al~\cite{abu2007comparison}, and Chandrasekaran et. al
~\cite{chandrasekaran2006phishing}.

Since phishing attacks mostly require victims to click on a link, the URLs 
and the corresponding webpages also provide us clues from which we can 
determine whether it is phish related. For example, in early days, many
phish URL will have IP address in it, like \texttt{http://135.12.44.20/index.php}. This feature is extremely rare for
legitimate websites. Another characteristic is that phish domains tend to
relatively long, containing multiple segments. For example, 9794.myonlineaccounts2.abbeynational.co.uk.syrialand.com. The idea is that
a very long domain can confuse people and sometimes people only read the 
beginning of a domain and think it is legitimate~\cite{wu2006security}.
Whittaker et. al. from Google has experimented with multiple URL related
features and implemented them in their machine learning algorithm
~\cite{whittaker2010large}. They also used other URL metadata like PageRank
to assist the detection. Another source to look at is the phishing
webpage. Phishing attacks usually try to mimic other trusted websites and
make people think they are visiting the original legitimate websites. They
also frequently ask people to type their credentials or provide payment
information, since the goal of the attackers is to steal these information.
Researchers have used the content of the webpages as the feature to further
enhance their algorithms. Zhang et. al.~\cite{zhang2007cantina} has used 
the word frequency on the webpage(TF-IDF) to classify the webpages. Wardman 
et. al.~\cite{wardman2008automating} compared the content on target webpage
with other popular webpages to see if the webpages in question are trying
to mimic those legitimate ones. All these resources provide additional
confidence to people when deciding if a URL is a phish URL.


\section{Threat Intelligence Sharing}
How much cyber-threat one can capture is largely depend on the scope of
vantage points he has on the Internet. Organizations with higher scope of 
observation,
like bigger Internet telescope or larger number of customers, tend to
capture more threat. However, no entity has the capability to monitor
the entire Internet, and with so many activities going on in the cyber
world these days, each individual entities is only able to monitor a
tiny fraction of what is happening. 

This encouraged \textit{Threat Intelligence Sharing}, where different 
entities collect threat information individually and share the data
with each other, so every one will get a higher coverage on potential 
threat. Many threat intelligence providers are offering the platform
for threat intelligence sharing, including IBM X-Force 
Exchange~\cite{ibmxforce}, AlienVault Open Threat 
Exchange~\cite{alienvaultotx}, Facebook 
ThreatExchange~\cite{facebookthreatexchange} etc. These platforms
enable companies (not necessarily security companies), organizations 
or individual security researchers to contribute threat intelligence 
they collected. Companies are also forming alliance and exchange their
intelligence data within the group, like Cyber Threat
Alliance~\cite{cyberthreatalliance}.

One problem during threat intelligence sharing is data specification.
Since different entities could collect data in different ways and
also record the data in different format. Without a clear unified
data format, the data being shared will provide little benefit to
the recipients, since they will not be able to understand and utilize
the data. This is a nontrivial task, since this unified data format
has to indicate clearly what does the data ``mean''. If one entity
just shares a list IPs and claims these IPs are malicious IPs, it
does not help much since it is crucial for the recipient to know
exactly why they are malicious, e.g, because they are massively 
scanning the Internet, trying to brute-force log in to SSH servers,
or they serve as C2 servers for malware. Since companies have different 
security hygiene, they will use the data differently based on its
meanings. So it is critical to specify clearly the meaning of the
data during sharing. Because of this, standard threat intelligence 
formats have been proposed and developed, 
notably IODEF~\cite{IODEF}, CybOX~\cite{CybOX} and STIX~\cite{STIX}, 
that try to standardize the threat intelligence presentation and sharing. 

\section{Threat Intelligence Data Analysis}

Several studies have examined the effectiveness of blacklist-based 
threat intelligence~\cite{kuhrer2014paint, ramachandran2006revealing, 
ramachandran2007filtering, sheng2009empirical, sinha2008shades}.
Ramachandran~\etal~\cite{ramachandran2007filtering} showed that spam 
blacklists are both incomplete (missing 35\% of the source IPs of 
spam emails captured in two spam traps), and slow in responding 
(20\% of the spammers remain unlisted after 30 days).
Sinha~\etal~\cite{sinha2008shades} further confirmed this result by 
showing that four major spam blacklists have very high false negative
rates, and analyzed the possible causes of the low coverage.
Sheng~\etal~\cite{sheng2009empirical} studied the effectiveness of
phishing blacklists, showing the lists are slow in reacting to
highly transient phishing campaigns.

Other studies have analyzed the general attributes of threat
intelligence data. Pitsillidis~\etal~\cite{tasters:imc12} studied the
characteristics of spam domain feeds, showing different perspectives
of spam feeds, and demonstrated that different feeds are suitable for
answering different questions. Thomas~\etal~\cite{thomas2016abuse}
constructed their own threat intelligence by aggregating the abuse
traffic received from six Google services, showing a lack of
intersection and correlation among these different sources. 

The limitations of the previous measurement works are that these 
studies tend to only focused on specific types of threat intelligence 
sources, like spam or phish blacklists, and they only evaluated one 
aspect of the data characteristic, like the operational performance, 
rather than generalize the measurement and define threat intelligence 
metrics that can be extended beyond the work.

Little work before had defines a general measurement methodology to
examine threat intelligence across a broad set of types and categories.
Metcalf~\etal~\cite{metcalf2015blacklist} collected and measured IP
and domain blacklists from multiple sources, but again only focused 
on volume and intersection analysis. One missing piece in these works
is that they did not approach the problem from the perspective of 
consumers of Threat Intelligence.  After all, it is the consumers that
will support this industry, and research communities should look more
into their needs. This is one major motivation of my work, which will
be discussed in Chapter~\ref{chapter:data_character}.
\section{Threat Intelligence Uses}
\label{sec:threat_intel_uses}

Threat intelligence data, at a high-level, promises that by compiling up-to-date 
information about known threats (i.e., IP addresses, domain names, file hashes, 
etc.), recipients of the data will be able to better defend their systems from 
future attacks. Therefore, the primary use cases of threat intelligence is 
network defense. Intrusion detection systems or firewalls can directly put 
the data in the system and block the corresponding IP or DNS traffic. Popular 
open source projects like Snort~\cite{snortids}, Zeek~\cite{zeekids} all provide 
these functionalities. Commercial products like Palo Alto Network
firewall~\cite{paloaltofirewall}, Fortinet firewall~\cite{fortinetfirewall}, 
Cisco firewall~\cite{ciscofirewall} also incorporate threat intelligence data in 
their defense systems. 

Besides directly taking action, threat intelligence can also be used in security
monitoring and post forensic analysis. In these cases, the system raises alarms
when there are matches between network activities and threat intelligence data.
Threat intelligence data usually come with confidence and severity level for each
individual data item. When investigating these alarms, administrators can 
prioritize the investigation based on the severity level. The system that in 
charge of collecting and organization security alarms are called Security 
Information and Event Management system, or SIEM system. Popular SIEM systems
include Splunk~\cite{splunk}, Sumo Logic~\cite{sumologic} and LogRhythm NextGen 
SIEM~\cite{logrhythm} etc.

In academic research, threat intelligence data is usually used as extra source 
data to assist their study, or to evaluate the performance of their systems or
algorithms. For example, Hao et. al.~\cite{hao2016predator} explored using the
characteristic domains during domain registration to detect potential malicious
domains. In the study, the authors use Spamhaus domain blacklist, URIBL to check 
the accuracy of their prediction. Singh et. al.~\cite{singh2017characterizing} 
studied the characteristic of Tor exit blocking in the wild, and use public and
private threat intelligence sources to see how much of Tor exit IPs are listed
on these sources. These are experimental use cases researchers have explored 
with threat intelligence data. 

One question that has not been investigated is how threat intelligence 
products are actually being used by organizations currently in the industry.
Understanding the real-world use cases gives us ideas about the adoption of threat
intelligence data, which is essential information to know in the threat
intelligence ecosystem. More importantly, the usage of different products offers
us insight into the potential impact they could cost on the Internet. As 
discussed in Chapter~\ref{chapter:data_character}, false positives are relatively 
common in threat intelligence feeds. If an organization is using one threat
intelligence IP feed in its firewall for IP-based traffic blocking, and there
is a false positive(a benign IP address) in this feed, then all the users in
that organization will be affected. From another side, if one online host is
added to an IP feed mistakenly, then this host will lose access to all the
organizations that use this feed as a blocking ruleset. Therefore, the actual
usage of the data in industry is a crucial topic that should get more attention
from the security community.

But this problem is also a very challenging problem. The primary challenge 
is that there are many ways people can use these data. It can be directly 
used to block network traffic, or just raise an alarm in their Security
Information and Event Management(SIEM) systems. Some use cases do not even 
have a well-defined behavior that we can quantitatively measure. 
The actions derived from threat intelligence data can also happen at different
network layers. For example, an organization can deny access on network layer; 
it can also deny access on application layer, like HTTP 403 Forbidden. 
This diverse possibility of use cases makes it hard to assess this problem as 
a whole. 

Little work has been done to try and understand how threat
intelligence data is being used on a large scale. The only work that tries to
look at threat intelligence data from this perspective are industrial surveys
-- wherein organizations fill out questionnaires. One such survey conducted by
the Ponemon Institute~\cite{ponemon2018cti}, surveyed 1,200 IT and IT
security practitioners, asking if they use threat intelligence products, and
if they do what tools do they use that utilize the threat intelligence data.
The survey also asked their user experience. SANS Institute did a
similar study~\cite{shackleford2017cyber} where they surveyed 600
participants from a diverse industry background and asked questions about
their threat intelligence usage. These works are limited in terms
of scale, and their results are all at a very high-level. Although they offered
some useful insight, they can not provide us a concrete understanding
of threat intelligence uses on a large scale. My measurement,
which will be discussed in Chapter~\ref{chapter:data_usage},
is the first work that systematically looks at the problem of inferring
threat intelligence uses and explore its implications.

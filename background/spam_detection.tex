\subsection{Spam Detection}
Spam email, also referred as unsolicited bulk email or junk mail, is 
Internet mail that is sent to a group of recipients who have no intention 
to receive it. They are particularly harmful, as these emails jams users' 
mailboxes, engulf important personal mail, waste network bandwidth and 
can even crash mail-servers. Spam emails also serve as an important
way of advertising products in underground market, like prescription drugs, 
illegal porn, replica of other brands etc. It is an old yet still popular
threat on the Internet. Threat intelligence spam data contains email address 
used by the spammers, domains and IP addresses where spammers' mail servers 
are located. Organizations, and even ISPs, tend to use these data to block 
the incoming mail traffic.

Email providers, ISPs or anti-spam organizations usually set up 
\textit{Spamtrap} to capture spam emails. Spamtrap are the email addresses
that are created not for communication, but rather to lure spam. These
email addresses do not belong to any person and will not involve in any
kind of communication. These addresses will not be revealed openly online,
so only unsolicited spammers, who tend to collect target email addresses
by crawling the Internet, or go through all possible lexical combinations
for email names, will hit these addresses. The Spamtrap here serve as a 
bait to capture spammers. People also recycle long out-of-date email 
addresses as Spamtrap addresses. 

However, simply regarding all emails received in Spamtrap as spam emails 
will create a substantial amount of false positives, since legitimate 
senders with poor data hygiene or acquisition practices end up hit the traps 
as well. To further distinguish spam emails and consequently identify the
senders, one needs to look at the content of emails themselves. Numerous 
statistical algorithms have been proposed by researchers to filter spam
emails from legitimate ones. At its core, spam filtering can be viewed as
a text categorization task: given the full text content of an email, 
decide whether it is spam email or benign email. A variety of supervised
machine learning techniques have been tested for spam filtering. Like the
naive Bayes classifier~\cite{androutsopoulos2000evaluation, sahami1998bayesian, schneider2003comparison}, 
RIPPER rule induction algorithm~\cite{cohen1996learning},
Support Vector Machine~\cite{drucker1999support}, memory-based learning
~\cite{androutsopoulos2000learning}, AdaBoost~\cite{carreras2001boosting},
and maximum entropy model~\cite{zhang2003filtering}. These algorithms all
convert email headers and body into features for the machine learning model, 
the ``bag-of-words'' approach,
and different feature reduction and weight assignment strategy have been
explored. These models can all achieve decent accuracy, and have been tested
and deployed in real-world. 

Email providers also get help from the customers to identify spams. Since
customers can mark an email as spam manually, having a large customer base
enables the email provider to collect a large amount of spam emails with high
accuracy, and therefore track down the domains and IP addresses used by the 
spammers.
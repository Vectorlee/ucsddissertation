\subsection{Compare with Previous Method}

At a high level, the objective of the measurement is to determine, from a third
point, if a {\reflector} blocks a blacklist IP. The blocking behavior that we
want to measure is \textit{inbound blocking} -- that is the incoming traffic
is blocked and as a result traffic does not reach the intended host. A
typical example is a network firewall, where it can stop certain network
packets from reaching hosts behind the firewall. Here we abstract the
{\reflector} as $Host_A$, blacklist IP as $Host_B$, and the problem is to
detect the connectivity between two hosts on the Internet.

%Another type of blocking is %\textit{outbound blocking}, where the outgoing
%traffic to a ``blocked host'' from a host within the network is blocked. %A
%typical example is network censorship. Our measurement in this paper only
%focuses on inbound blocking, %and we will explain more about this in the
%following sections.

\begin{figure}[t]
\centering
\includegraphics[width=0.8\columnwidth]{data_usage/images/croped_method_old.pdf}
\caption{Measurement method used in previous work.}
\label{fig:old_method}
\end{figure}

In previous works~\cite{pearce2017augur, ensafi2014detecting}, to measure
the connectivity between two hosts from a third party, they send spoofed packets
to impersonate one of the hosts. This methodology -- one we refer
to as the \textit{triangle measurement} takes advantage of the TCP 3-way
handshake protocol, as shown in Figure~\ref{fig:old_method}.

The measurement machine first sends a probe to the $Host_A$, in this case a TCP
SYN-ACK packet. $Host_A$ responds with a RST packet since it received a SYN-ACK
without the preceding SYN packet. Thus, the measurement machine gets the
first IP ID $IP\mhyphen ID_1$ from the RST packet (corresponds to Step 1 in
the figure). Next, the measurement machine sends a spoofed TCP SYN packet to
$Host_B$, with source IP address set to the IP address of $Host_A$ (Step 2).
$Host_B$ then sends a responding SYN-ACK packet to $Host_A$, which causes $Host_A$
to respond with a RST packet (Step 3), and increment its IP ID counter by 1.
Finally, the measurement machine probe $Host_A$ again and get the second IP ID
$IP\mhyphen ID_2$ (Step 4).

Now we can infer whether $Host_A$ is inbound blocking traffic from $Host_B$ by
observing the difference between $IP\mhyphen ID_1$ and $IP\mhyphen ID_2$.
Assuming there is no packet loss, and that $Host_A$ does not have any extra
traffic besides our measurement traffic, then
$IP\mhyphen ID_2 = IP\mhyphen ID_1 + 2$ implies there is no inbound blocking,
since it indicates that $Host_A$ received both packets in step 3 and step 4,
while $IP\mhyphen ID_2 = IP\mhyphen ID_1 + 1$ means there is blocking.

Previous work chose this ``triangle measurement'' because it ensures that in
step 3, the packets from $Host_B$ will go through the same routes as the
traffic originated from $Host_B$. So from $Host_A$'s perspective, it can not
identify that the traffic from $Host_B$ are spoofed. However, in this schema,
one hard requirement is that $Host_B$ needs to be active and responding to TCP
SYN probe. This is not an issue for censorship measurement, as $Host_B$ in this
case are popular sites (Google, Facebook, Twitter etc.) that guaranteed will
respond to SYN probe.

However, in our case, we need to measure whether $Host_A$ is blocking traffic
from a blacklist IP($Host_B$). But there is no guarantee that these IP
addresses are active and thus may not respond to our SYN probe. In fact, we
found that the percentage of responding IPs in a blacklist can be as low as less
than 20\%. This dramatically reduces the candidate IPs we can sample from a
blacklist to test, especially for small blacklists that only have a few hundred
IPs. Furthermore, there are many other additional constraints when shortlisting
IPs for measurement from a blacklist.
The requirement that blacklist IPs respond to SYN probe does not work for our
use case. 
%\textcolor{red}{Explain the choice for only inbound blocking}

In order to get around this limitation we adjust the measurement methodology.
In this new methodology, the measurement machine directly sends spoofed
packets to the target host, as shown in Figure~\ref{fig:new_method}. In
this case, the measurement machine first probes $Host_A$ to get the first IP ID
$IP\mhyphen ID_1$, then it sends a spoofed packet, with source IP set to
$Host_B$(Blacklist IP), directly to $Host_A$. Finally, it sends a second probe
to $Host_A$ and get the second IP ID $IP\mhyphen ID_2$. Now we can use the same
logic as before to infer whether $Host_A$ is inbound blocking $Host_B$. In this
approach, we do not require $Host_B$ to be actively responding SYN packets, any
IP address can be used here to conduct the test. The drawback is that the
spoofed packets now at times go through a different route versus the packets
originated from $Host_B$. Some network that implement spoofed packet
detection~\cite{ferguson2000rfc2827} could drop our spoofed packets, giving us
a false signal of inbound blocking. Therefore, when selecting hosts we conduct
extensive tests to weed out hosts that have such detection logic in place. We
find that not a lot of target hosts have such detection logic. We will talk in
detail about host selection in the follow sections.

%Therefore, we modified the measurement so that the measurement machine  directly
%send spoofed packets to host A, as shown in Figure~\ref{fig:new_method}. In
%this case, the measurement machine first probes host A to get the first IP ID
%\texttt{IP\_ID\_1}, then it sends a spoofed packet, with source IP as host B,
%directly to A. Finally, it sends a second probe to host A and get the second
%IP ID \texttt{IP\_ID\_2}. Now we can use the same logic as before to infer
%whether host A is inbound blocking host B. In this approach, we do not
%require host B to be actively responding SYN packets, any IP address can be
%used here to conduct the test. The drawback now is that the spoofed packets
%will likely go through a different route versus the packets originated from
%host B. Some network that implemented spoofed packet detection~\cite{ferguson2000rfc2827}
%could drop our spoofed packets, giving us a false
%signal of inbound blocking. Therefore, when selecting hosts for the
%measurement, we conduct extensive tests to make sure the hosts do not have
%such detection logic in place. Our experiment also shows that not a lot of
%hosts have such detection logic. We will talk in detail about host selection
%in Section~\ref{sec:requirement_host}.

\begin{figure}[t]
\centering
\includegraphics[width=0.8\columnwidth]{data_usage/images/croped_method_new.pdf}
\caption{Measurement method used in this work.}
\label{fig:new_method}
\end{figure}
\begin{abstract}
  This paper investigates a simple question: to what extent are
  Internet hosts proactively blocking traffic known to be associated
  with security threats?  We design and implement an inference technique, inspired
  by recent work on censorship measurement, to test if a remote host
  is blocking traffic from a given IP address.  Using this
  technique, we measure over {\reflroughnum} U.S. hosts
  and test whether they consistently block connections with IPs
  identified on popular IP blacklists. Together, we explore the use of public blacklists
  on the Internet, and demonstrate the evidence for more widespread use of
  other blacklists for traffic blocking. Our measurement shows that about
  one third of the hosts we surveyed enforce security related traffic blocking.




%Threat Intelligence, both as a concept and a product, has been increasingly
%gaining prominence in the security industry. There are now hundreds of
%vendors offering their threat intelligence solutions as a mix of public and
%commercial products. Given the rising use of threat intelligence there has
%been a concerted effort to understand the data that backs these threat
%intelligence products and also improve the associated systems and sharing
%protocols.

%However, there is an important aspect of the threat intelligence ecosystem
%that has not been well studied -- the use of threat intelligence products by
%organizations at a large scale. Understanding the adoption of threat
%intelligence is crucial since it not only offers us the holistic view of how
%the security industry is moving but also gives us critical insight into the
%potential impact of these threat intelligence feeds: once there are false
%positives in the data, how many organizations would be affected.

%The understanding of the
%usage of threat intelligence feeds is now even more pertinent given recent
%research that shows threat intelligence feeds differ greatly and tend to have
%a large number of false positives. Thus, studying the adoption of threat
%intelligence will allow us to not only gauge the prevalence of threat
%intelligence feeds but also the potential collateral damage these threat
%intelligence feeds may pose.

%In this paper, we take a first look at inferring the use of threat intelligence.
%Specifically, we focus on one class of threat intelligence use: IP-based network
%traffic blocking. This work details the measurement methodology and the result
%of a large scale experiment across {\reflroughnum} hosts in the United States, and
%provides the first concrete picture about the usage of a diverse set of popular
%public threat intelligence feeds in the United States. We also look at the
%how organizations update their threat intelligence feeds and the consistency of
%blocking behavior within the network. Our results show that although using
%particular feeds to block traffic is not very common, the security related
%network blocking in general is relatively popular.

%Threat Intelligence, both as a concept and as a product, is becoming more and
%more popular in the security industry in recent years. Hundreds of public and
%commercial vendors are offering their threat intelligence data solutions, and
%many research works had looked into this field, trying to understand the data
%and improve the associated systems and sharing protocols.

%However, there is one important aspect of threat intelligence that has not been
%well studied: how these data have been used by different users on a large scale.
%Understanding the usage of threat intelligence is an essential problem, as it
%not only offers us the big picture of how people are adopting threat intelligence,
%but also gives us critical insight into the potential impact of these data: If
%there are false positives in the data, how many people would be affected, and what
%would the collateral damage be. The lack of understanding on usage is a crucial
%missing piece in threat intelligence research.

%In this paper, we take the first look into this problem by focusing on one specific
%way of using threat intelligence data: IP-based traffic blocking. We designed the
%measurement methodology and conducted a large scale experiment over 280K online
%hosts in the United States. We checked these hosts against a diverse set of popular
%public Threat Intelligence IP feeds, and provide the first concrete picture about
%how many hosts are using these data, how quickly do they update and the
%consistency of blocking behavior within networks.
%\textcolor{red}{Last sentence will be the result of the large scale analysis.}
%We also looked into the prevalence of blacklist based traffic blocking at a high-level.
\end{abstract}

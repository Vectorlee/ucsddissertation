\section{Discussion}

In this chapter, I implement and test a technique for inferring
the deployment of network layer blacklisting and, further, for
attributing the use of particular blacklists in particular networks.
There are a range of limitations in this pilot study, most
significantly including potential selection bias arising from using
quiescent U.S. hosts running older versions of Windows. This is a 
limitation of the methodology itself. 
Another limitation lies in the fact that I exclusive use public 
blacklist data (i.e., since I do not have access to high-priced 
commercial threat intelligence feeds which could be distinct).

However, even given these limitations, the measurements of
220K hosts reveal a number of interesting artifacts.  First, I
discovered the widespread use of \emph{some} kind of network layer
blocking (affecting over a third of hosts in the data set) even if it
is not consistent with membership in any of the lists I track.
This demonstrated the prevalence of security related blocking 
even among machines with low security hygiene. Most most previous 
network disruption measurement explain the reason as censorship 
(whether nation wide or at corporate level). My work highlights 
the prevalence of security related blocking, which serves as a
reminder that for all the following network connectivity study,
researchers should always keep in mind the possibility of security
related blocking.

Second, I find that there is evidence of intra-network diversity in
traffic blocking policy. While a number of network prefixes have
consistent blocking behavior across multiple hosts, quite a few do
not, suggesting different network security policies are being employed
on different subnets. This implies that when measuring network 
behaviors, researchers should not assume the consistency within a 
network.

Finally, for blacklist use that can be precisely attributed the 
most widely used blacklists (Spamhaus DROP and eDROP and DShield Top) 
are also those that have extremely low false positives.
~\footnote{The DROP and eDROP lists are a small subset
of Spamhaus' feed that specifically deals with address for which the
entire network prefix is believed to be abusive (e.g., prefix
hijacking).} This suggests that for many networks proactive traffic
blocking is gated on having lists of sufficient accuracy to remove
the risks of accidentally blocking legitimate traffic.

\section{Partial Blocking}
\label{sec:partial-blocking}

%% In Section~\ref{sec:perfect-blocking}, we looked into {\reflectors} that
%% consistently block all blacklist IPs sampled for a blacklist. We further
%% confirm this by re-running the measurement three times with three different
%% set of blacklist IPs sampled from the blacklist at different times. However,
%% our measurement also discovers blocking behavior that is not ``perfect''.
%% Every time we sample blacklist IPs and test, many {\reflectors} only block
%% some of the IPs we sampled, while leaving the rest of IPs unblocked. We refer
%% to these cases as \textit{partial blocking}. In this section, we dive deeper
%% into these cases of partial blocking. Specifically, we look at different reasons
%% one may see partial blocking and then look in more detail at {\reflectors}
%% that show consistent, and significant partial blocking.

When performing our experimental runs we noticed that a small
percentage of reflectors consistently blocked a significant subset of
blacklist IPs, but not all, in \textit{every experiment}.
This consistency suggests that, while the
{\reflector} may not use the exact blacklist, there is a large overlap
between the blacklist and the blocking policy of the {\reflector}.  We
refer to such reflectors as exhibiting \textit{significant partial blocking}
behavior.  Figure~\ref{fig:reflector-breakdown} shows these reflectors
are just 0.4\% of all reflectors that we tested, but they still
correspond to 21\% of the number of reflectors that perfectly block at
least one blacklist and therefore motivate further investigation.  As
a result, in this section we characterize this partial blocking
behavior in more detail.

%% Therefore, if a {\reflector} only blocks a small portion of the IPs we
%% sampled from a blacklist, it could not give us much insight about the
%% host's behavior regarding that blacklist.

%% However, we observe that many of these {\reflectors} show partial
%% blocking behavior consistently: Every time we test with a certain
%% blacklist, they always block a significant portion of the IPs we
%% sampled from that list.

%% However, before we dive deeper into these {\reflectors} we briefly touch upon
%% geo-blocking -- a significant contributor to partial blocking and how we try and
%% reduce the effects that geo-blocking may have on our results.
%\textcolor{red}{Show the table about the breakdown between >75\% and >50\%}.

\subsection{Geo-Blocking}
\label{sec:geo-blocking}

%% Geo-blocking is the case where the {\reflector} blocks all the traffic from a
%% certain country. Some sites apply this blocking either for policy reason
%% (e.g. block GDPR countries~\cite{bbcnews}.), or for security reasons (e.g.
%% block countries where recent attacks originated from, like Russia or China).
%% Although during blacklist IP sampling, we go the extra mile to increase the
%% geo diversity of IPs, we might still end up with sampled IPs largely from a
%% few countries. Primarily, this is because the blacklist itself can have a
%% disproportionate amount of IPs from only a few countries. For example,
%% {\dshieldtop} on Jan. 25th had over 50\% of its IPs coming from Netherlands.
%% If a host blocks traffic from Netherlands, then we would miss conclude that
%% host is partially blocking {\dshieldtop}.

%% To eliminate the effects of geo-blocking, we conducted an additional
%% experiment to identify the geo-blocking hosts among our {\reflectors}. For
%% each country the blacklist IPs had covered, we randomly select 20 IPs from
%% that country, and test against all {\reflectors}, using the same IP ID
%% technique, to identify if the hosts that are blocking these IPs. Here we
%% acquire IP address prefixes of each country from 4 location services:
%% MaxMind~\cite{maxmind}, IP2Location~\cite{ip2location}, IPDeny~\cite{ipdeny}
%% and IPIP.net~\cite{ipip}, and we only sample IPs from the intersection of the
%% prefixes provided by each service. Put differently, for each country we test,
%% we only sample from the IPs where all 4 location services are agree they are
%% from that country. We further make sure the sampled IPs are not overlapped
%% with any of the blacklists we have, and they are BGP routable. We tested 20
%% countries in total, from big countries like Russia and China, to small island
%% countries like Singapore and Seychelles.

%% In total, we identify 677 {\reflectors} that at least block one country we
%% tested, and 473 {\reflectors} that block at least two countries. 549
%% {\reflectors} block IPs from China, making it the most popular target of
%% geo-blocking in our collection. Russia is the second most blocked country,
%% where we found 414 hosts blocking Russian IPs, followed by Hong Kong and
%% Vietnam, for which there are 196 and 193 {\reflectors} blocking their IPs
%% respectively. European countries like Belgium, Netherlands and France also
%% have over 60 hosts blocking them.

%% Note that what we discovered here is network layer geo-blocking, and there
%% can be other forms of geo-blocking, like application layer blocking (e.g.
%% HTTP 403 Forbidden). We do not try to comprehensively cover all possible
%% geo-blocking cases since this experiment was only to help us identify the
%% potential side-effects of geo-blocking in our experiment, so we could better
%% distinguish the blacklist related blocking behavior in our experiment.

%% One potential cause of partial blocking is geo-blocking.

Geo-blocking is one type of blocking we identified that contributes to
this partial blocking.  A {\reflector} using geo-blocking will
drop all traffic from a particular country.  Organizations typically
use geo-blocking either for policy reason (e.g., block GDPR
countries~\cite{bbcnews}), or for security reasons (e.g., block
countries that are sources of attacks, such as Russia or China).  If a
reflector uses geo-blocking, we will observe it blocking IPs on a
blacklist if those IPs happen to be located in a blocked country.  Although
we take extra efforts to increase the geo diversity when sampling IPs(Section~\ref{sec:methtarg}),
this kind of overlap can still be exacerbated if a blacklist happens to have
concentrations of IPs from particular countries.  For example,
{\dshieldtop} on January 25, 2020 had over 50\% of its IPs geo-located
in the Netherlands.  If a reflector blocks traffic from the
Netherlands, then we would observe that the reflector is partially
blocking {\dshieldtop}.

%% Recall that Although during blacklist IP sampling, we go the extra
%% mile to increase the geo diversity of IPs, we might still end up with
%% sampled IPs largely from a few countries. Primarily, this is because
%% the blacklist itself can have a disproportionate amount of IPs from
%% only a few countries.

To identify whether a {\reflector} uses geo-blocking, we test whether
the {\reflector} consistently blocks a set of IPs from a particular
country.  For all countries related to blacklist IPs that we test, we
first enumerate the IP address prefixes for those countries using four
IP-based location services: MaxMind~\cite{maxmind},
IP2Location~\cite{ip2location}, IPDeny~\cite{ipdeny}, and
IPIP.net~\cite{ipip}.  For each country, we then randomly select 20 IP
addresses from those prefixes such that: (1) all four location
services agree on the country label for the IPs, (2) the IPs do not
appear on a blacklist, and (3) the IPs are BGP routable.  Then for all
{\reflectors}, we test whether it blocks all of the randomly-chosen
IPs for each country.  If it does, then we conclude that it uses
geo-blocking.

We tested our {\reflectors} against 20 countries in total, ranging
from large countries like Russia and China to small island countries
like Singapore and Seychelles.  Overall, only a small number of
{\reflectors}, 614 (0.28\%), block at least one country, and 432 block
at least two.  China is blocked most often, with 501 of the 614
{\reflectors} blocking random IPs in China.  Russia is second at 376,
followed by Hong Kong (177) and Vietnam (175).  European countries
including Belgium, Netherlands, and France also have over 60
{\reflectors} blocking them.

Note that our methodology identifies geo-blocking at the network
layer.  Other forms of geo-blocking exist, such as application-layer
blocking (e.g., HTTP 403 Forbidden).  We do not explore all possible
geo-blocking mechanisms since our goal was to identify {\reflectors}
using geo-blocking at the network layer.

%% \noteby{GV}{Thinking about it more, if we had more time we could turn
%%   geo-blocking into a result instead of treating it as error.  The
%%   challenge with fitting it into the current framing is that it is not
%%   always significant partial blocking.}

%% , as they could introduce noise
%% when determining network-level blocking based upon blacklists.

\subsection{Significant Partial Blocking}

\begin{table}[t]
\centering
\small
%\begin{adjustbox}{width=\columnwidth}
\begin{tabular}{l r r r | r r r }
 \toprule
 \textbf{}           &\multicolumn{3}{c}{\textbf{Blocked over 75\%}}    &\multicolumn{3}{c}{\textbf{Blocked over 50\%}} \\
 \textbf{Blacklist}  &\textbf{Hosts}   &\textbf{/24s}   &\textbf{ASes}  &\textbf{Hosts}   &\textbf{/24s}   &\textbf{ASes} \\
 \midrule
 DROP      & 28      & 18     & 3   &  23      & 21    & 3\\
 eDROP     & 96      & 60     & 32  &  49      & 27    & 18\\
 DTop        & 157     & 66     & 19  &  319     & 165   & 72\\
 ET     & 13      & 7      & 6   &  31      & 19    & 17\\
 BDS           & 8       & 5      & 5   &  7       & 7     & 7\\
 Feodo             & 65      & 30     & 19  &  23      & 17    & 15\\
 Snort       & 11      & 9      & 7   &  34      & 20    & 17\\
 DE       & 148     & 38     & 1   &  13      & 11    & 4\\
 Tor             & 63      & 35     & 26  &  31      & 19    & 16\\
 %% {\spamhausdrop}      & 28      & 18     & 3   &  23      & 21    & 3\\
 %% {\spamhausedrop}     & 96      & 60     & 32  &  49      & 27    & 18\\
 %% {\dshieldtop}        & 157     & 66     & 19  &  319     & 165   & 72\\
 %% {\etcompromised}     & 13      & 7      & 6   &  31      & 19    & 17\\
 %% {\bdsatif}           & 8       & 5      & 5   &  7       & 7     & 7\\
 %% {\feodo}             & 65      & 30     & 19  &  23      & 17    & 15\\
 %% {\snortfilter}       & 11      & 9      & 7   &  34      & 20    & 17\\
 %% {\blocklistde}       & 148     & 38     & 1   &  13      & 11    & 4\\
 %% {\ettor}             & 63      & 35     & 26  &  31      & 19    & 16\\
 \midrule
 \textbf{Total}              & 492     & 207    & 71  & 459      & 257   & 108\\
 \bottomrule
\end{tabular}
%\end{adjustbox}
\caption{Number of reflectors exhibiting significant partial blocking on each blacklist.}
\label{tab:partial-blocking-reflectors}
\end{table}

%\noteby{GV}{At what point in the workflow are geo-blocking reflectors
%  removed?}

%% Recall that we selected all {\reflectors} exclusively within the US to
%% minimize blocking behavior caused by censorship
%% (Section~\ref{sec:methrefl}).

In addition to geo-blocking, there are reasons why a {\reflector} may
block a blacklist IP that is not due to the reflector using that
blacklist.  A host may have internal policies that deny access from
some network providers, or network administrators may add IPs into
their firewall on an ad-hoc basis based on an organization's internal
strategies or policies.  These alternate blocking behaviors could
overlap with the blacklist IPs we sampled, leading to partial blocking
behavior in our results.

%% We go through all partial blocking we observed, and remove the ones
%% that caused by geo-blocking.

Having identified reflectors using geo-blocking, we remove these
reflectors from further consideration.  We then calculate two groups of
{\reflectors}: for each blacklist, we identify {\reflectors} that
partially block over 75\% of our sampled IPs \textit{every time} we
test them, and another group where they partially block over 50\% of
our sampled IPs.
%We exclude the cases where we fail to conclude the blocking behavior of a
%few {\reflectors} and blacklist IP pairs (after 15 trials)
%\noteby{GV}{Not clear on this?}.

%Every time we test, we sample IPs from the exclusive part of that blacklist,
%where we check against over 100 public IP blacklist (including some obsolete
%ones). Therefore,

For each blacklist, Table~\ref{tab:partial-blocking-reflectors}
summarizes the number of {\reflectors} that fall into each category.
If a {\reflector} is blocking more than 75\% of our sampled IPs every
time, it is plausible that the {\reflector} is using a source that is
very similar to the blacklist we tested.  For instance, there could be
other blacklists where the data collection methodology is similar to
the method used by the public blacklists in our study.  Previous work
has shown that commercial products can aggregate data from public
blacklists, and that they potentially conduct post-processing to
eliminate some content~\cite{li2019reading}.

We suspect that the partial blocking behavior in
Table~\ref{tab:partial-blocking-reflectors} likely results from such
cases.  In particular, the number of {\reflectors} that are partial
blocking {\dshieldtop} is relatively high compared with other
blacklists---it covers over 30\% of all {\reflectors} in the first
group and over 69\% in the second group, suggesting that this list may
be relatively frequently included into other lists.

%% Looking into the organizations related to these partial blocking hosts
%% (Section~\ref{sec:perfect-blocking}), {\feodo} is again covered by the
%% most diverse set of organizations, showing its popularity among
%% different sectors.  \noteby{GV}{Keep in mind that the next section
%%   claims that we cannot map to organizations...}

%% Overall, though, only a small number of {\reflectors} show significant
%% partial blocking, which is consistent with the finding in the previous
%% section.

%% (Less is More)
%% One interesting fact is that there are 253 {\reflectors} from
%% the two partial blocking groups overlap with the {\reflectors} we
%% identified previously that use a blacklist, which is because some
%% {\reflectors} use one blacklist but only partially block another.

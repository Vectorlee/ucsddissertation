\section{Partial Blocking}
\label{sec:partial-blocking}

When performing the experimental runs I noticed that a small
percentage of reflectors consistently blocked a significant subset of
blacklist IPs, but not all, in \textit{every experiment}.
This consistency suggests that, while the
{\reflector} may not use the exact blacklist, there is a large overlap
between the blacklist and the blocking policy of the {\reflector}.  We
refer to such reflectors as exhibiting \textit{significant partial blocking}
behavior.  Figure~\ref{fig:reflector-breakdown} shows these reflectors
are just 0.4\% of all reflectors that I tested, but they still
correspond to 21\% of the number of reflectors that perfectly block at
least one blacklist and therefore motivate further investigation.  As
a result, in this section I characterize this partial blocking
behavior in more detail.

\subsection{Geo-Blocking}
\label{sec:geo-blocking}

Geo-blocking is one type of blocking I identified that contributes to
this partial blocking.  A {\reflector} using geo-blocking will
drop all traffic from a particular country.  Organizations typically
use geo-blocking either for policy reason (e.g., block GDPR
countries~\cite{bbcnews}), or for security reasons (e.g., block
countries that are sources of attacks, such as Russia or China).  If a
reflector uses geo-blocking, one will observe it blocking IPs on a
blacklist if those IPs happen to be located in a blocked country.  Although
I take extra efforts to increase the geo diversity when sampling IPs(Section~\ref{sec:methtarg}),
this kind of overlap can still be exacerbated if a blacklist happens to have
concentrations of IPs from particular countries.  For example,
{\dshieldtop} on January 25, 2020 had over 50\% of its IPs geo-located
in the Netherlands.  If a reflector blocks traffic from the
Netherlands, then I would observe that the reflector is partially
blocking {\dshieldtop}.

To identify whether a {\reflector} uses geo-blocking, I test whether
the {\reflector} consistently blocks a set of IPs from a particular
country.  For all countries related to blacklist IPs that I test, I
first enumerate the IP address prefixes for those countries using four
IP-based location services: MaxMind~\cite{maxmind},
IP2Location~\cite{ip2location}, IPDeny~\cite{ipdeny}, and
IPIP.net~\cite{ipip}.  For each country, I then randomly select 20 IP
addresses from those prefixes such that: (1) all four location
services agree on the country label for the IPs, (2) the IPs do not
appear on a blacklist, and (3) the IPs are BGP routable.  Then for all
{\reflectors}, I test whether it blocks all of the randomly-chosen
IPs for each country.  If it does, then I conclude that it uses
geo-blocking.

I tested the {\reflectors} against 20 countries in total, ranging
from large countries like Russia and China to small island countries
like Singapore and Seychelles.  Overall, only a small number of
{\reflectors}, 614 (0.28\%), block at least one country, and 432 block
at least two.  China is blocked most often, with 501 of the 614
{\reflectors} blocking random IPs in China.  Russia is second at 376,
followed by Hong Kong (177) and Vietnam (175).  European countries
including Belgium, Netherlands, and France also have over 60
{\reflectors} blocking them.

Note that the methodology identifies geo-blocking at the network
layer.  Other forms of geo-blocking exist, such as application-layer
blocking (e.g., HTTP 403 Forbidden).  I do not explore all possible
geo-blocking mechanisms since my goal was to identify {\reflectors}
using geo-blocking at the network layer.

\subsection{Significant Partial Blocking}

\begin{table}[t]
\centering
\caption{Number of reflectors exhibiting significant partial blocking on each blacklist.}
%\begin{adjustbox}{width=\columnwidth}
\begin{tabular}{l r r r | r r r }
 \toprule
 \textbf{}           &\multicolumn{3}{c}{\textbf{Blocked over 75\%}}    &\multicolumn{3}{c}{\textbf{Blocked over 50\%}} \\
 \textbf{Blacklist}  &\textbf{Hosts}   &\textbf{/24s}   &\textbf{ASes}  &\textbf{Hosts}   &\textbf{/24s}   &\textbf{ASes} \\
 \midrule
 DROP      & 28      & 18     & 3   &  23      & 21    & 3\\
 eDROP     & 96      & 60     & 32  &  49      & 27    & 18\\
 DTop        & 157     & 66     & 19  &  319     & 165   & 72\\
 ET     & 13      & 7      & 6   &  31      & 19    & 17\\
 BDS           & 8       & 5      & 5   &  7       & 7     & 7\\
 Feodo             & 65      & 30     & 19  &  23      & 17    & 15\\
 Snort       & 11      & 9      & 7   &  34      & 20    & 17\\
 DE       & 148     & 38     & 1   &  13      & 11    & 4\\
 Tor             & 63      & 35     & 26  &  31      & 19    & 16\\
 \midrule
 \textbf{Total}              & 492     & 207    & 71  & 459      & 257   & 108\\
 \bottomrule
\end{tabular}
%\end{adjustbox}

\label{tab:partial-blocking-reflectors}
\end{table}


In addition to geo-blocking, there are reasons why a {\reflector} may
block a blacklist IP that is not due to the reflector using that
blacklist.  A host may have internal policies that deny access from
some network providers, or network administrators may add IPs into
their firewall on an ad-hoc basis based on an organization's internal
strategies or policies.  These alternate blocking behaviors could
overlap with the blacklist IPs I sampled, leading to partial blocking
behavior in the results.

Having identified reflectors using geo-blocking, I remove these
reflectors from further consideration.  I then calculate two groups of
{\reflectors}: for each blacklist, I identify {\reflectors} that
partially block over 75\% of the sampled IPs \textit{every time} I
test them, and another group where they partially block over 50\% of
the sampled IPs.


For each blacklist, Table~\ref{tab:partial-blocking-reflectors}
summarizes the number of {\reflectors} that fall into each category.
If a {\reflector} is blocking more than 75\% of the sampled IPs every
time, it is plausible that the {\reflector} is using a source that is
very similar to the blacklist I tested.  For instance, there could be
other blacklists where the data collection methodology is similar to
the method used by the public blacklists in this study.  Previous work
has shown that commercial products can aggregate data from public
blacklists, and that they potentially conduct post-processing to
eliminate some content~\cite{li2019reading}.

I suspect that the partial blocking behavior in
Table~\ref{tab:partial-blocking-reflectors} likely results from such
cases.  In particular, the number of {\reflectors} that are partial
blocking {\dshieldtop} is relatively high compared with other
blacklists---it covers over 30\% of all {\reflectors} in the first
group and over 69\% in the second group, suggesting that this list may
be relatively frequently included into other lists.

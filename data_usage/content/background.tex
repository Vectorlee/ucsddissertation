\section{Background}
%\label{sec:background}

%\textcolor{red}{Industrial survey about threat intelligence usage}
\subsection{Threat Intelligence Study}

There is a large literature concerning the the use of various kinds of
``threat intelligence'' (not always using that term). One popular focus among
these is evaluating their effectiveness, including Sinha et
al.~\cite{sinha2008shades} who analyze coverage and accuracy of spam
blacklists and Pitsillidis et al.~\cite{tasters:imc12} who characterize and
compare a range of such popular lists. Sheng et al. perform similar analyses
on the effectiveness of phishing blacklists~\cite{sheng2009empirical} and
Kuhrer et al. explore this approach for malware
domains~\cite{kuhrer2014paint}. Still others have explored techniques to
better populate such lists, including Ramachandran et al's work on inferring
botnet IP addresses from DNSBL lookups~\cite{ramachandran2006revealing},
their subsequent work building ``behavioral'' spam
blacklists~\cite{ramachandran2007filtering} or the work of Hao et al. for
predicting future domain abuse~\cite{hao2016predator,hao2013understanding}
(among others). More recently, Thomas et al. explored the value of sharing
threat intelligence data across functional areas (e.g., mail spam, account
abuse, search abuse) and found limited overlap and significant numbers of
false positives when used in this manner~\cite{thomas2016abuse}. Many of
these results are echoed by Li et al. who empirically evaluated threat
intelligence feeds of same type for factors such as coverage and
accuracy~\cite{li2019reading}. Finally, there is also a literature focused on
meta issues in threat information sharing, including standardizing data
formats and associated sharing protocols~\cite{barnum2012standardizing,
wagner2016misp, mavroeidis2017cyber, burger2014taxonomy}.

However, there is comparatively little work focused on understanding how
threat intelligence data is being used in practice. Indeed, the literature
that exists is primary driven by
surveys~\cite{ponemon2018cti,shackleford2017cyber} and not validated by any
empirical measurement. It is this gap that drives our efforts in this paper.

\subsection{Internet Connection Blocking}
There has been significant empirical exploration of
Internet connection blocking in the setting of Internet freedom and access.
Indeed, there are a range of studies that measure connection block in the
context of Internet censorship~\cite{aryan2013internet,
park2010empirical,anderson2012splinternet,zittrain2003internet,clayton2006ignoring},
geo-blocking~\cite{opennetsurvey, mcdonald2018403,afroz2018exploring}, and
Tor blocking~\cite{singh2017characterizing, khattak2016you}.

Most of these studies rely on vantage points sited in the target
networks being studied, and so are not directly helpful in our work.
However, recent work by Ensafi et al~\cite{ensafi2014detecting} and
Pearce et al~\cite{pearce2017augur} has removed this requirement using
an indirect side channel technique to test connectivity between pairs
of remote hosts.  While our approach differs in a number of ways from
theirs, it is based on the same idea of using IP ID to infer remote
traffic sending.

%In these cases, countries or organizations block network access mostly out of
%policy reasons. In the censorship case, countries like China and Iran primarily
%targets on pornography sites, news sites, social media and politically sensitive
%sites~\cite{china2news, greatfire, aryan2013internet}. In the geo-blocking case,
%websites often block access from other countries because of legal restrictions~\cite{mcdonald2018403}.
%For example, U.S. export controls limit both physical and intellectual property
%that U.S.-based entities can transfer to some nationalities without explicit
%authorization~\cite{exportcontrol}, and GDPR has triggered several major
%U.S.-based news sites block access from Europe entirely~\cite{bbcnews}.

% First Paragraph: Explain what is IP ID
\subsection{IP ID Side Channel}
\label{sec:ipidchannel}
The IP ID traffic side channel has been well-known since at least the
mid 1990s.  In particular, the Identification (ID) field of an IPv4
packet is a 16-bit value in the IP packet header, designed to support
fragmentation by providing a unique value that can be used to group
packet fragments belonging to the same IP
datagram.~\cite{postel1981rfc0791}.  An easy way to ensure
``uniqueness'' is to populate the IP ID field using a per-host global
counter that increments after each packet is sent.  However, this
implementation choice has a side effect that the \emph{number} of
packets being sent is implicitly encoded in the \emph{change} in IP ID
over time.  Thus, by probing a host multiple times one can use the
value of the returned IP ID to infer how many packets have been sent
by the remote host \emph{between} the two probes.  This side channel
has been employed for a wide variety of measurement purposes,
including anonymous port scanning\cite{antirez1998}, host alias
detection~\cite{spring2002measuring} and enumerating hosts behind
NATs~\cite{bellovin2002technique} among others.  While most operating
systems no longer use such a simple approach, it is still reasonably
common across the Internet since all versions of Windows up to version
7 used the global increment algorithm.\footnote{In recent work, Klein
  and Pinkas reverse engineer the contemporary Windows IP ID algorithm
  and develop an attack to create collisions among a set of
  counters~\cite{klein2019ip}.  We did not explore using this attack
  in our pilot study.}

% Second paragraph: how a lot of people are implementing it.
%One easy way to implement the IP ID field for networking is to use a global
%counter. In this case, a system uses one 16-bit variable to set the IP ID
%value for all the IP packets it sends out, and increments the variable by 1
%after every packet. This simple solution ensures the IP ID value of all
%packets are unique, and it is how many early systems implemented their
%IP ID mechanisms~\cite{klein2019ip}.

%This implementation create a side channel that allows anyone without
%access to a host to observe the traffic volume from that host. An observer,
%by probing the host twice separated by some time interval and checking the
%corresponding IP ID increase, can learn about the number of packets the host
%sent out during this period. This side channel can be used to observe many
%different network effects, like

%In order to eliminate this side channel, new systems implement the IP ID
%generation by using different counters for different traffic, so different
%observers will see a different IP ID sequences from the same host. However,
%there are still a significant number of hosts on the Internet that still use
%the global counter implementation. For example, Windows 8 and older still use
%this implementation~\cite{klein2019ip}. These hosts give us sizable amount of
%candidates to conduct our measurement.

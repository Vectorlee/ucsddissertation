\section{Technical Background}

\subsection{Internet Connection Blocking}
In this work, I focus on one specific way of how organizations are using 
threat intelligence data: use threat intelligence IP feeds as direct rule-set 
to block network traffic. This behavior is very similar to other forms of
Internet connection blocking, most notable examples are Internet censorship,
geo-blocking and Tor blocking.

Many previous work has studied Internet censorship~\cite{aryan2013internet,
park2010empirical,anderson2012splinternet,zittrain2003internet,
clayton2006ignoring}, geo-blocking~\cite{opennetsurvey, mcdonald2018403,afroz2018exploring}, 
and Tor blocking~\cite{singh2017characterizing, khattak2016you}. However, 
these measurement studies all rely on having vantage points in the target 
regions, so the researchers can directly measure the network effects and 
acquire the results. There are serval public projects that facility such 
studies by providing access points around the global, like RIPE
Atlas~\cite{ripeatlas}, OONI~\cite{ooni} and ICLab~\cite{iclab}.
For Tor blocking, it is also not hard for the researchers to get access 
to a Tor exit node~\cite{khattak2016you}, then conduct the following 
measurement from that node. But these studies are restrained by the vantage
points they could get, as it is hard to get these vantage points in large
quantities, and these points are also heavily biased towards certain networks.
For country-wide censorship or geo-blocking measurement, this is not a big
issue. But in this case, I want to conduct large scale measurements over a 
broad set of online hosts, it is very difficult to acquire vantage points 
that can meet my requirements.

Recent work by Ensafi et al~\cite{ensafi2014detecting} and Pearce et 
al~\cite{pearce2017augur}. demonstrated the method to use IP ID side 
channel to measure Internet connectivity. This is an indirect method 
and allow an observer to measurement the connectivity between two hosts 
without having access to either of the hosts. In this study, I do not 
have access to either threat intelligence feeds, or large amount of 
online hosts, this method is an ideal method for this measurement. But 
I need to modify the original method to better suit this experiment, 
and I also need to take extra effort to eliminate the interfering 
signals. I will establish on the methodology in more detail in
Section~\ref{sec:methodology}.

% First Paragraph: Explain what is IP ID
\subsection{IP ID Side Channel}
\label{sec:ipidchannel}
The Identification (ID) field of an IPv4 packet is a 16-bit value in the IP
packet header. It is designed primarily to support IP packet fragmentation
and reassembly. If a large IP packet is fragmented when sent through the
network, the receiving end will use this field to identify the fragments from
the same IP packet and re-assemble them. This requires the 16-bit value to be
unique for every datagram of a given source address, destination address, and
protocol, such that it does not repeat within the maximum datagram
lifetime~\cite{postel1981rfc0791}.

% Second paragraph: how a lot of people are implementing it.
One easy way to implement the IP ID field for networking is to use a global
counter. In this case, a system uses one 16-bit variable to set the IP ID
value for all the IP packets it sends out, and increments the variable by 1
after every packet. This simple solution ensures the IP ID value of all
packets are unique, and it is how many early systems implemented their
IP ID mechanisms~\cite{klein2019ip}.

This implementation create a side channel that allows anyone without
access to a host to observe the traffic volume from that host. An observer,
by probing the host twice separated by some time interval and checking the
corresponding IP ID increase, can learn about the number of packets the host
sent out during this period. This side channel can be used to observe many
different network effects, like host alias detection~\cite{spring2002measuring},
where observers can detect if two online hosts actually correspond to one 
physical machine by monitoring their IP ID changes, and NATed host 
counting~\cite{bellovin2002technique}, where observers can identify the 
number of machines behind a NAT by tracking the number of IP ID sequences 
from the packets comming out of the network.

In order to eliminate this side channel, new systems implement the IP ID
generation by using different counters for different traffic, so different
observers will see a different IP ID sequences from the same host. However,
there are still a significant number of hosts on the Internet that still use
the global counter implementation. For example, Windows 8 and older still use
this implementation~\cite{klein2019ip}. These hosts give us sizable amount of
candidates to conduct this measurement.
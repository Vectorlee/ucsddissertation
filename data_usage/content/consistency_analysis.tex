\section{Blocking Consistency}
\label{sec:consistency}
% TODO: Location of the Table! :(
% GEOFF: Link to the breakdown by AS Type
% https://docs.google.com/spreadsheets/d/1QAWr1qb01reJVAmhvl-ukOJjDl5E_1kr_uIQrAOwWyo/edit#gid=691238820
%
%% \begin{table*}[t]
%% 	\centering
%% 	\small
%% 	\begin{tabular}{l|rr|r|r} \toprule
%% 		\multicolumn{1}{c}{\textbf{}} & \multicolumn{2}{c}{\textbf{Consistent}}                                                      & \multicolumn{1}{c}{\textbf{Almost Consistent}} & \multicolumn{1}{c}{\textbf{Inconsistent}}      \\
%% 		\textbf{Blacklist}            & \multicolumn{1}{c}{\textbf{No Blocking (\%)}} & \multicolumn{1}{c}{\textbf{Consistent (\%)}} & \multicolumn{1}{c}{\textbf{Off By One (\%)}}   & \multicolumn{1}{c}{\textbf{Inconsistent (\%)}} \\ \hline
%% 		\textbf{\bdsatif}              & 35540 (88.19\%)                               & 398 (0.99\%)                                 & 2332 (5.79\%)                                  & 2030 (5.03\%)                                  \\
%% 		\textbf{\blocklistde}          & 40228 (99.82\%)                               & 24 (0.06\%)                                  & 13 (0.03\%)                                    & 35 (0.09\%)                                    \\
%% 		\textbf{\dshieldtop}           & 39603 (98.27\%)                               & 393 (0.98\%)                                 & 138 (0.34\%)                                   & 166 (0.41\%)                                   \\
%% 		\textbf{\etcompromised}        & 39535 (98.10\%)                               & 298 (0.74\%)                                 & 102 (0.25\%)                                   & 365 (0.91\%)                                   \\
%% 		\textbf{\feodo}                & 39943 (99.11\%)                               & 195 (0.48\%)                                 & 42 (0.10\%)                                    & 120 (0.31\%)                                   \\
%% 		\textbf{\snortfilter}          & 39830 (98.83\%)                               & 250 (0.62\%)                                 & 88 (0.22\%)                                    & 132 (0.33\%)                                   \\
%% 		\textbf{\spamhausdrop}         & 39320 (97.57\%)                               & 750 (1.86\%)                                 & 30 (0.07\%)                                    & 200 (0.50\%)                                   \\
%% 		\textbf{\spamhausedrop}        & 39792 (98.73\%)                               & 349 (0.87\%)                                 & 36 (0.09\%)                                    & 123 (0.31\%)                                   \\
%% 		\textbf{\ettor}                & 40000 (99.25\%)                               & 165 (0.41\%)                                 & 43 (0.11\%)                                    & 92 (0.23\%)                                    \\
%% 		\midrule
%% 		\textbf{Overall} 			   & 353791 (97.54\%)   & 2822	(0.78\%) & 2824 (0.78\%) & 3263 (0.90\%) \\
%% 		\midrule
%% 		\textbf{Control Group}  	   & 40255  (99.89\%)	& 3	(0.01\%)	& 37 (0.09\%)	 & 5 (0.01\%)	   \\
%% 		\bottomrule
%% 	\end{tabular}
%% 	\caption{Consistency Breakdown}
%% 	\label{tab:consistency-breakdown}
%% \end{table*}

As a final analysis, we explore the consistency of reflector blocking
behavior at a coarser granularity.  A common use case of blacklists is
at the granularity of an organization, often via some kind of network
appliance.  In such a scenario, we would expect the blocking behavior
of {\reflectors} to be consistent across an organization: if one
{\reflector} blocks a blacklist IP, then other {\reflectors} in the
same organization should also block it.

Ideally we would like to map reflectors to organizations to answer
this question.  However, mapping an IP to an organization is a
challenging problem, particularly with the increasing use of WHOIS
anonymization.  Instead, we use the common, more tractable technique
of aggregating reflectors by their /24 prefix.  As a result, in this
section we answer a methodological question: If we aggregate
reflectors by their /24 prefix, do the aggregated reflectors exhibit
consistent blocking behavior?  Is the /24 prefix aggregation a useful
proxy for expected consistent blocking by organizations?

%% However, mapping an IP to an organization is a
%% challenging problem, particularly with the increasing use of WHOIS
%% anonymization.  Further, multiple ``organizations'' could potentially
%% map to a single AS, as with transit and access ASes. Thus, we use the
%% common method of assuming hosts on the same /24 prefix are part of the
%% same network and, as such, the same organization.

%% Specifically, for each /24 with more than one {\reflector}, we check
%% if the {\reflectors} block the exact same set of sampled blacklist IPs.

%\subsection{Consistency Methodology}

Our data set has 134,370 {\reflectors} that are part of /24s with more
than one {\reflector}.  For each blacklist, we categorize the blocking
behavior of multiple {\reflectors} in the same /24 into one of three
categories: \textit{consistent}, \textit{almost consistent}, and
\textit{inconsistent}.  We define a /24 to be ``consistent'' for a
blacklist if \textit{all} the individual {\reflectors} in that /24
block the \textit{exact same} blacklist IPs.
%
A /24 is ``almost consistent'' if the blocking behavior of the
{\reflectors} in a /24 differs only by one IP. For example, a /24 is
``almost consistent'' if it has four {\reflectors}, three of which
block the same 21 IPs from a blacklist, and the fourth {\reflector}
blocks 20 out the same 21 IPs.
%% since if we had tried more than 15 times we could have potentially
%% found them to be consistent.
%
Finally, we consider all other /24s ``inconsistent''.

%% any /24 that does not fall into the aforementioned two categories,
%% we consider them as ``inconsistent''.

%there is the case of missed blacklist IPs -- blacklist IPs for which we could
%not get a good signal for blocking behavior. In the specific instance, where
%the sum of ``perfect'' and ``missed'' blacklist IPs is exactly the same for
%all individual {\reflectors} in a /24 we carve out that /24 as unknown since
%had we persisted beyond 15 tries we could have found it as consistent.

%\subsection{Consistency Results}

Using these definitions, Table~\ref{tab:consistency-breakdown} shows
the consistency results for all the /24s that have more than one
{\reflector}.  The results are dominated by /24s that do not show any
blocking behavior.  We consider such /24s consistent since all the
hosts under these /24s block the same number (zero) of blacklist IPs,
but these cases also do not provide much insight.

Excluding the ``no blocking'' cases, then in the presence of any
blocking, consistency of blocking behavior at a /24 granularity is far
from guaranteed.  As discussed in Section~\ref{subsec:fpfn_analysis},
our measurement technique has very low false positive and false
negative rates.  Measurement error can potentially explain some
``almost consistent'' cases and perhaps some ``inconsistent'' cases,
too.  However, the consistency results for the control group,
comprised of 20 randomly sampled US IPs
(Section~\ref{sec:perfect-blocking}), shows that the potential effect
of measurement error on consistency is small.  In other words, the
inconsistent cases do indeed demonstrate different blocking behavior
among hosts within the same /24.

One situation that could lead to inconsistent blocking behavior within
a /24 is when the network belongs to a cloud or hosting provider, and
the IPs within the same /24 are used by distinct entities.  For
instance, when manually examining the inconsistent /24s for {\bdsatif}
(which has the highest inconsistency), we found more than 60\% of
these /24s belong to cloud or hosting providers.  Another situation
leading to inconsistent block behavior is when a /24 belongs to an ISP
which sub-allocates IP addresses to different customers.

In summary, our results indicate that we cannot assume consistent
blocking behavior for {\reflectors} in the same /24 network.

\section{Blocking Consistency}
\label{sec:consistency}

\begin{table*}[t]
	\centering
    \caption{The blocking consistency of the reflectors in /24 prefixes with more than one reflector.}
	\begin{tabular}{l|rr|r|r} \toprule
		\multicolumn{1}{c}{\textbf{}} & \multicolumn{2}{c}{\textbf{Consistent}}                                                      & \multicolumn{1}{c}{\textbf{Almost Consistent}} & \multicolumn{1}{c}{\textbf{Inconsistent}}      \\
		\textbf{Blacklist}            & \multicolumn{1}{c}{\textbf{No Blocking (\%)}} & \multicolumn{1}{c|}{\textbf{Consistent (\%)}} & \multicolumn{1}{c|}{\textbf{Off By One (\%)}}   & \multicolumn{1}{c}{\textbf{Inconsistent (\%)}} \\
\midrule
{\bdsatif} & 35,540 \hspace*{2pt} (88.19\%) & 398 \hspace*{2pt} (0.99\%) & 2,332 \hspace*{2pt} (5.79\%) & 2,030 \hspace*{2pt} (5.03\%) \\
{\blocklistde} & 40,228 \hspace*{2pt} (99.82\%) & 24 \hspace*{2pt} (0.06\%) & 13 \hspace*{2pt} (0.03\%) & 35 \hspace*{2pt} (0.09\%) \\
{\dshieldtop} & 39,603 \hspace*{2pt} (98.27\%) & 393 \hspace*{2pt} (0.98\%) & 138 \hspace*{2pt} (0.34\%) & 166 \hspace*{2pt} (0.41\%) \\
{\etcompromised} & 39,535 \hspace*{2pt} (98.10\%) & 298 \hspace*{2pt} (0.74\%) & 102 \hspace*{2pt} (0.25\%) & 365 \hspace*{2pt} (0.91\%) \\
{\feodo} & 39,943 \hspace*{2pt} (99.11\%) & 195 \hspace*{2pt} (0.48\%) & 42 \hspace*{2pt} (0.10\%) & 120 \hspace*{2pt} (0.31\%) \\
{\snortfilter} & 39,830 \hspace*{2pt} (98.83\%) & 250 \hspace*{2pt} (0.62\%) & 88 \hspace*{2pt} (0.22\%) & 132 \hspace*{2pt} (0.33\%) \\
{\spamhausdrop} & 39,320 \hspace*{2pt} (97.57\%) & 750 \hspace*{2pt} (1.86\%) & 30 \hspace*{2pt} (0.07\%) & 200 \hspace*{2pt} (0.50\%) \\
{\spamhausedrop} & 39,792 \hspace*{2pt} (98.73\%) & 349 \hspace*{2pt} (0.87\%) & 36 \hspace*{2pt} (0.09\%) & 123 \hspace*{2pt} (0.31\%) \\
{\ettor} & 40,000 \hspace*{2pt} (99.25\%) & 165 \hspace*{2pt} (0.41\%) & 43 \hspace*{2pt} (0.11\%) & 92 \hspace*{2pt} (0.23\%) \\
\midrule
\textbf{Overall} & 353,791 \hspace*{2pt} (97.54\%) & 2,822 \hspace*{2pt} (0.78\%) & 2,824 \hspace*{2pt} (0.78\%) & 3,263 \hspace*{2pt} (0.90\%) \\
\midrule
\textbf{Control Group} & 40,255 \hspace*{2pt} (99.89\%)	& 3 \hspace*{2pt} (0.01\%) & 37 \hspace*{2pt} (0.09\%) & 5 \hspace*{2pt} (0.01\%) \\
\bottomrule
	\end{tabular}
	\label{tab:consistency-breakdown}
\end{table*}

As a final analysis, I explore the consistency of reflector blocking
behavior at a coarser granularity.  A common use case of blacklists is
at the granularity of an organization, often via some kind of network
appliance.  In such a scenario, I would expect the blocking behavior
of {\reflectors} to be consistent across an organization: if one
{\reflector} blocks a blacklist IP, then other {\reflectors} in the
same organization should also block it.

Ideally I would like to map reflectors to organizations to answer
this question.  However, mapping an IP to an organization is a
challenging problem, particularly with the increasing use of WHOIS
anonymization.  Instead, I use the common, more tractable technique
of aggregating reflectors by their /24 prefix.  As a result, in this
section I answer a methodological question: If I aggregate
reflectors by their /24 prefix, do the aggregated reflectors exhibit
consistent blocking behavior?  Is the /24 prefix aggregation a useful
proxy for expected consistent blocking by organizations?

My data set has 134,370 {\reflectors} that are part of /24s with more
than one {\reflector}.  For each blacklist, I categorize the blocking
behavior of multiple {\reflectors} in the same /24 into one of three
categories: \textit{consistent}, \textit{almost consistent}, and
\textit{inconsistent}.  I define a /24 to be ``consistent'' for a
blacklist if \textit{all} the individual {\reflectors} in that /24
block the \textit{exact same} blacklist IPs.
%
A /24 is ``almost consistent'' if the blocking behavior of the
{\reflectors} in a /24 differs only by one IP. For example, a /24 is
``almost consistent'' if it has four {\reflectors}, three of which
block the same 21 IPs from a blacklist, and the fourth {\reflector}
blocks 20 out the same 21 IPs.
Finally, I consider all other /24s ``inconsistent''.


Using these definitions, Table~\ref{tab:consistency-breakdown} shows
the consistency results for all the /24s that have more than one
{\reflector}. The results are dominated by /24s that do not show any
blocking behavior.  I consider such /24s consistent since all the
hosts under these /24s block the same number (zero) of blacklist IPs,
but these cases also do not provide much insight.

Excluding the ``no blocking'' cases, then in the presence of any
blocking, consistency of blocking behavior at a /24 granularity is far
from guaranteed.  As discussed in Section~\ref{subsec:fpfn_analysis},
the measurement technique has very low false positive and false
negative rates.  Measurement error can potentially explain some
``almost consistent'' cases and perhaps some ``inconsistent'' cases,
too.  However, the consistency results for the control group,
comprised of 20 randomly sampled US IPs
(Section~\ref{sec:perfect-blocking}), shows that the potential effect
of measurement error on consistency is small.  In other words, the
inconsistent cases do indeed demonstrate different blocking behavior
among hosts within the same /24.

One situation that could lead to inconsistent blocking behavior within
a /24 is when the network belongs to a cloud or hosting provider, and
the IPs within the same /24 are used by distinct entities.  For
instance, when manually examining the inconsistent /24s for {\bdsatif}
(which has the highest inconsistency), I found more than 60\% of
these /24s belong to cloud or hosting providers.  Another situation
leading to inconsistent block behavior is when a /24 belongs to an ISP
which sub-allocates IP addresses to different customers.

In summary, the results indicate that one cannot assume consistent
blocking behavior for {\reflectors} in the same /24 network.

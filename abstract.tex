\begin{dissertationabstract}

Threat Intelligence, both as a concept and a product, has been increasingly
gaining prominence in the security industry. At a high-level, it is the 
``knowledge'' that helps organizations understand and mitigate cyber-attacks.
Most commonly, it refers to the collection of threat indicators---IP 
addresses, domain names, file hashes, etc. known to be associated with 
attacks. By compiling and distributing this information, it is believed 
that recipients will be able to better 
defend their systems from future attacks. Thus, there are now hundreds of
vendors offering their threat intelligence solutions as a mix of public and
commercial products. 

However, our understanding of this data, its characterization, and the
extent to which it can meaningfully support its intended uses, is
still quite limited. Furthermore, how the data is being used by
organizations, how popular it is, and what impact it could have on the 
Internet are also not clear to our community. We lack an empirical 
assessment of real-world threat intelligence, both in terms of the data
itself and its usage, and it is important to first understand the current 
status of threat intelligence, then can we reasonably discuss how to make
improvements.

In this dissertation, I take an empirical approach to study threat 
intelligence and try to address these gaps. In particular, I explore 
this topic from two perspectives: 1) Studying the characteristics of 
threat intelligence data itself and 2) Exploring how they are used in the 
real-world. In particular, I formally defined a set of metrics for 
analyzing threat intelligence data feeds and use these measures to 
systematically evaluate a broad range of public and commercial feeds. 
Further, I ground my quantitative assessments using external measurements 
to investigate issues of coverage and accuracy. Finally, I
designed a method using the IP ID side channel to test if a remote host 
is blocking traffic from a given IP address. Using this technique, I 
measured over {\reflroughnum} U.S. hosts and tested whether they 
consistently block connections with IPs identified on popular IP blacklists. 
Beyond these blacklists, I also demonstrate the evidence for more widespread 
use of security related traffic blocking. Together, my work provides
an in-depth look into the current picture of threat intelligence and augments
the knowledge of our community on this topic.

\end{dissertationabstract}

\begin{dissertationabstract}

Threat Intelligence, both as a concept and a product, has been increasingly
gaining prominence in the security industry. At a high-level, it is the 
``knowledge'' that helps organizations understand and mitigate cyber-attacks.
Most commonly, it refers to the collection of known threats---IP addresses, 
domain names, file hashes, etc. that known to be associated with attacks.
By compiling up-to-date information about these known threats, threat 
intelligence data promises that the recipients will be able to better 
defend their systems from future attacks. Thus, there are now hundreds of
vendors offering their threat intelligence solutions as a mix of public and
commercial products. 

However, our understanding of this data, its characterization and the
extent to which it can meaningfully support its intended uses, is
still quite limited. Furthermore, how the data is being used by
organizations, how popular it is and what impact it could have 
on the Internet are also not clean to our community. 
It is important to first understand the current status of threat 
intelligence, then can we reasonably talk about how to make improvements.

In this dissertation, I take an empirical approach to study threat 
intelligence. In particular, I explore this topic from two perspectives:
the characteristic of data itself and how people are using it in the real-world.
In the data characteristic study, I formally defined a set of metrics for 
analyzing threat intelligence data feeds and using these measures to 
systematically characterize a broad range of public and commercial feeds. 
Further, I ground my quantitative assessments using external measurements 
to qualitatively investigate issues of coverage and accuracy. In the study
of real-world uses, I designed a method using the IP ID side channel to test 
if a remote host is blocking traffic from a given IP address. Using this
technique, I measured over {\reflroughnum} U.S. hosts and test whether they 
consistently block connections with IPs identified on popular IP blacklists. 
Beyond these blacklists, I also demonstrated the evidence for more widespread 
use of other blacklists for traffic blocking. Together, my work provides
an in-depth look into the current status of threat intelligence and augment
the knowledge of our community on this topic.

\end{dissertationabstract}

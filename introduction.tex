% Optional Introduction
\begin{dissertationintroduction}
Computer security is an inherently adversarial discipline in which
each ``side'' seeks to exploit the assumptions and limitations of the
other.  Attackers rely on exploiting knowledge of vulnerabilities,
configuration errors or operational lapses in order to penetrate
targeted systems, while defenders in turn seek to improve their
resistance to such attacks by better understanding the nature of
contemporary threats and the technical fingerprints left by attacker's
craft.  Invariably, this means that attackers are driven to innovate
and diversify while defenders, in response, must continually monitor
for such changes and update their operational security practices
accordingly.  This dynamic is present in virtually every aspect of the
operational security landscape, from anti-virus signatures to the
configuration of firewalls and intrusion detection systems to incident
response and triage.  Common to all such reifications, however, is the
process of monitoring for new data on attacker behavior and using that
data to update defenses and security practices. Indeed, the extent to
which a defender is able to gather and analyze such data effectively
defines a de facto window of vulnerability---the time during which an
organization is less effective in addressing attacks due to ignorance
of current attacker behaviors.

This abstract problem has given rise to a concrete demand for
contemporary threat data sources that are frequently collectively
referred to as \emph{threat intelligence} (\ti).  By far the most
common form of such data are so-called \emph{indicators of
  compromise:} simple observable behaviors that signal that a host or
network may be compromised.  These include both network indicators
such as IP addresses (e.g., addresses known to launch particular
attacks or host command-and-control sites, etc.) and file hashes
(e.g., indicating a file or executable known to be associated with a
particular variety of malware).  The presence of such indicators is a
symptom that alerts an organization to a problem, and part of an
organization's defenses might reasonably include monitoring its assets
for such indicators to detect and mitigate potential compromises as
they occur.

\verb!\mainmatter! macro because it should start on page~1.
\end{dissertationintroduction}

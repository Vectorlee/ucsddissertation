% Your fancy acks here. Keep in mind you need to ack each paper you
% use. See the examples here. In addition, each chapter ack needs to
% be repeated at the end of the relevant chapter.
\begin{acknowledgements}
First and foremost, I would like to thank my advisors, professor Stefan
Savage and professor Kirill Levchenko. When I first joined the PhD program,
I was an introvert student with little understanding of research. I was
too scared to try something new or to think boldly. Thanks to my advisors,
who both have very clean visions and are not afraid to take risks, I had a 
very exciting PhD experience. We tried a new idea for malware analysis, 
reverse-engineered Volkswagen engine firmware to try to disclose its emission 
cheating logic, and eventually moved to Threat Intelligence and try to understand 
the landscape of threat on the Internet. Some projects worked out, some did not, 
but I learned so much from this journey. They taught me to always follow my
interests, not the trend, when choosing projects, and never be afraid of 
failure. This attitude towards research had changed me a lot, both in the
way of doing research, also in the way of making life decisions. Besides, 
I really appreciate that both of my advisors gave me great freedom during
my study, which makes my PhD a lot less stressful, and at the same 
time taught me to be independent very early on. It is very fortunate for me to 
have their support and guidance on this journey.

I also liked to thank professor Geoffrey M. Voelker. Although he is not my
official advisor, he helped me a lot during two of my projects, both of 
which are covered in this dissertation. He also gave me a lot of valuable 
guidance during some of my lowest points. I really thank his kindness
and caring. I would also like to thank Professors Deian Stefan and Xinyu Zhang 
for being on my doctoral committee and being available whenever I needed help.

PhD experience is fun but also challenging, and I can not make it 
without the help from so many people, especially from my ``Foundry'' lab
mates (CSE 3142), where everyone is so kind and supportive.
Our lab has a very free, open atmosphere, and we discuss almost anything
freely here. With these discussions and exchange of ideas, I become 
more objective on thinking, and learned a lot on many different 
topics. We are also hard-working and spent many long nights together
at lab. I will never forget those times when we laugh and fight together. 
I would also like to thank members from ``Tiger'' lab (CSE 3152), we 
hang out frequently and give support to each other during hard times. The style 
of how they approach research problems---being very strict on defining the
problem and validating with real-world examples---inspired me a lot.

Special thanks to Danny Huang and Tianyin Xu. These two senior students(now 
both become professors) had helped me a lot during my PhD and give me
valuable advice. During my lowest points, I had multiple long conversations
with them, and they shared a lot of their own stories and encouraged me to
keep going. Their encouragement and the experience they shared is critical 
for my journey.

I would like than all the member of Sysnet group, including all the faculties,
staffs and students. It has been a wonderful 6 years for me, and I had many 
interactions with many people during this process, in security lunch, 
syslunch, sysnet hiking, CNS meeting etc. All these activities not only enable
me to learn a lot of stuff, but also make my journey much more enjoyable, and 
give me a strong feeling of belonging. Special thanks to Cindy Moore, who 
helped me numerous times setting up servers and fixing network issues. 

Chapter~\ref{chapter:data_character}, in part, is a reprint of the material 
as it appears in Proceedings of the Usenix Security 2019. \textit{Reading the 
Tea Leaves: A Comparative Analysis of Threat Intelligence.} Vector Guo Li,
Matthew Dunn, Paul Pearce, Damon McCoy, Geoffrey M. Voelker, Stefan Savage,
Kirill Levchenko. The dissertation author was the primary investigator and 
author of this paper. I really like to thank my coauthors on this paper,
without who I will never be able to finish this work. I like to especially 
thank Paul Pearce, who helped me a lot throughout the project. At the
beginning, when I had little idea about what to do and was so unfamiliar with 
data analysis tools, he gave me critical hands-on assists. I am also very
grateful to Alberto Dainotti and Alistair King for sharing the UCSD telescope
data and helping me with the analysis, also professor Micheal Bailey for 
sharing the Mirai Botnet data. Besides, I like to thanks professor Aaron 
Schulman, who encouraged me a lot when the first submission of this paper 
was rejected, and helped me regain my confidence in the work.

Chapter~\ref{chapter:data_usage}, in part, is a reprint of the material as 
submitted to the Proceedings of the Usenix Security 2020. \textit{
Clairvoyance: Inferring Blacklist Use on the Internet} Vector Guo Li, 
Gautam Akiwate, Yihui Chen, Geoffrey M. Voelker, Kirill Levchenko, Stefan 
Savage. The dissertation author was the primary investigator and author of 
this paper. I really appreciate the help from my coauthors, especially 
Gautam Akiwate. He helped me a lot on data collection and analysis,
and I really learned a lot from the way he looks at data problems. 
Furthermore, he gave me crucial moral support during some critical times
of the project. I will not be able to finish this work without his help.
I also really appreciate the suggestions I received from kc Claffy and 
Alex Gantman regarding this project.

Last but not least, I want to thank my parents. My parents know little
about research, but they know how to do things. They gave me lots of 
suggestions on my study, but I was too fool to realize how valuable those 
suggestions are. This journey is tough, and many times when I did not
know what to do, when I lost my hope, when I thought about giving up,
they supported me, encouraged me, gave me strength and hope to 
carry on and keep going. If the PhD journey is walking through a dark 
tunnel, then they are the torch in my hand that are always there,
calm me down, show me the way and bring me warmth. They are the source of 
my strength and my motivation to work hard. It is very lucky for me to have 
such wise and supportive parents, and I will never forget their love 
and caring.
\end{acknowledgements}
